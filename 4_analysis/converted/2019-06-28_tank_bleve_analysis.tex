

% A latex document created by ipypublish
% outline: ipypublish.templates.outline_schemas/latex_outline.latex.j2
% with segments:
% - standard-standard_packages: with standard nbconvert packages
% - standard-standard_definitions: with standard nbconvert definitions
% - ipypublish-doc_article: with the main ipypublish article setup
% - ipypublish-front_pages: with the main ipypublish title and contents page setup
% - ipypublish-biblio_natbib: with the main ipypublish bibliography
% - ipypublish-contents_output: with the main ipypublish content
% - ipypublish-contents_framed_code: with the input code wrapped and framed
% - ipypublish-glossary: with the main ipypublish glossary
%
%%%%%%%%%%%% DOCCLASS

\documentclass[10pt,parskip=half,
toc=sectionentrywithdots,
bibliography=totocnumbered,
captions=tableheading,numbers=noendperiod]{scrartcl}

%%%%%%%%%%%%

%%%%%%%%%%%% PACKAGES

\usepackage[T1]{fontenc} % Nicer default font (+ math font) than Computer Modern for most use cases
\usepackage{mathpazo}
\usepackage{graphicx}
\usepackage[skip=3pt]{caption}
\usepackage{adjustbox} % Used to constrain images to a maximum size
\usepackage[table]{xcolor} % Allow colors to be defined
\usepackage{enumerate} % Needed for markdown enumerations to work
\usepackage{amsmath} % Equations
\usepackage{amssymb} % Equations
\usepackage{textcomp} % defines textquotesingle
% Hack from http://tex.stackexchange.com/a/47451/13684:
\AtBeginDocument{%
    \def\PYZsq{\textquotesingle}% Upright quotes in Pygmentized code
}
\usepackage{upquote} % Upright quotes for verbatim code
\usepackage{eurosym} % defines \euro
\usepackage[mathletters]{ucs} % Extended unicode (utf-8) support
\usepackage[utf8x]{inputenc} % Allow utf-8 characters in the tex document
\usepackage{fancyvrb} % verbatim replacement that allows latex
\usepackage{grffile} % extends the file name processing of package graphics
                        % to support a larger range
% The hyperref package gives us a pdf with properly built
% internal navigation ('pdf bookmarks' for the table of contents,
% internal cross-reference links, web links for URLs, etc.)
\usepackage{hyperref}
\usepackage{longtable} % longtable support required by pandoc >1.10
\usepackage{booktabs}  % table support for pandoc > 1.12.2
\usepackage[inline]{enumitem} % IRkernel/repr support (it uses the enumerate* environment)
\usepackage[normalem]{ulem} % ulem is needed to support strikethroughs (\sout)
                            % normalem makes italics be italics, not underlines

\usepackage{translations}
\usepackage{microtype} % improves the spacing between words and letters
\usepackage{placeins} % placement of figures
% could use \usepackage[section]{placeins} but placing in subsection in command section
% Places the float at precisely the location in the LaTeX code (with H)
\usepackage{float}
\usepackage[colorinlistoftodos,obeyFinal,textwidth=.8in]{todonotes} % to mark to-dos
% number figures, tables and equations by section
% fix for new versions of texlive (see https://tex.stackexchange.com/a/425603/107738)
\let\counterwithout\relax
\let\counterwithin\relax
\usepackage{chngcntr}
% header/footer
\usepackage[footsepline=0.25pt]{scrlayer-scrpage}

% bibliography formatting
\usepackage[numbers, square, super, sort&compress]{natbib}
% hyperlink doi's
\usepackage{doi}

    % define a code float
    \usepackage{newfloat} % to define a new float types
    \DeclareFloatingEnvironment[
        fileext=frm,placement={!ht},
        within=section,name=Code]{codecell}
    \DeclareFloatingEnvironment[
        fileext=frm,placement={!ht},
        within=section,name=Text]{textcell}
    \DeclareFloatingEnvironment[
        fileext=frm,placement={!ht},
        within=section,name=Text]{errorcell}

    \usepackage{listings} % a package for wrapping code in a box
    \usepackage[framemethod=tikz]{mdframed} % to fram code

%%%%%%%%%%%%

%%%%%%%%%%%% DEFINITIONS

% Pygments definitions

\makeatletter
\def\PY@reset{\let\PY@it=\relax \let\PY@bf=\relax%
    \let\PY@ul=\relax \let\PY@tc=\relax%
    \let\PY@bc=\relax \let\PY@ff=\relax}
\def\PY@tok#1{\csname PY@tok@#1\endcsname}
\def\PY@toks#1+{\ifx\relax#1\empty\else%
    \PY@tok{#1}\expandafter\PY@toks\fi}
\def\PY@do#1{\PY@bc{\PY@tc{\PY@ul{%
    \PY@it{\PY@bf{\PY@ff{#1}}}}}}}
\def\PY#1#2{\PY@reset\PY@toks#1+\relax+\PY@do{#2}}

\expandafter\def\csname PY@tok@w\endcsname{\def\PY@tc##1{\textcolor[rgb]{0.73,0.73,0.73}{##1}}}
\expandafter\def\csname PY@tok@c\endcsname{\let\PY@it=\textit\def\PY@tc##1{\textcolor[rgb]{0.25,0.50,0.50}{##1}}}
\expandafter\def\csname PY@tok@cp\endcsname{\def\PY@tc##1{\textcolor[rgb]{0.74,0.48,0.00}{##1}}}
\expandafter\def\csname PY@tok@k\endcsname{\let\PY@bf=\textbf\def\PY@tc##1{\textcolor[rgb]{0.00,0.50,0.00}{##1}}}
\expandafter\def\csname PY@tok@kp\endcsname{\def\PY@tc##1{\textcolor[rgb]{0.00,0.50,0.00}{##1}}}
\expandafter\def\csname PY@tok@kt\endcsname{\def\PY@tc##1{\textcolor[rgb]{0.69,0.00,0.25}{##1}}}
\expandafter\def\csname PY@tok@o\endcsname{\def\PY@tc##1{\textcolor[rgb]{0.40,0.40,0.40}{##1}}}
\expandafter\def\csname PY@tok@ow\endcsname{\let\PY@bf=\textbf\def\PY@tc##1{\textcolor[rgb]{0.67,0.13,1.00}{##1}}}
\expandafter\def\csname PY@tok@nb\endcsname{\def\PY@tc##1{\textcolor[rgb]{0.00,0.50,0.00}{##1}}}
\expandafter\def\csname PY@tok@nf\endcsname{\def\PY@tc##1{\textcolor[rgb]{0.00,0.00,1.00}{##1}}}
\expandafter\def\csname PY@tok@nc\endcsname{\let\PY@bf=\textbf\def\PY@tc##1{\textcolor[rgb]{0.00,0.00,1.00}{##1}}}
\expandafter\def\csname PY@tok@nn\endcsname{\let\PY@bf=\textbf\def\PY@tc##1{\textcolor[rgb]{0.00,0.00,1.00}{##1}}}
\expandafter\def\csname PY@tok@ne\endcsname{\let\PY@bf=\textbf\def\PY@tc##1{\textcolor[rgb]{0.82,0.25,0.23}{##1}}}
\expandafter\def\csname PY@tok@nv\endcsname{\def\PY@tc##1{\textcolor[rgb]{0.10,0.09,0.49}{##1}}}
\expandafter\def\csname PY@tok@no\endcsname{\def\PY@tc##1{\textcolor[rgb]{0.53,0.00,0.00}{##1}}}
\expandafter\def\csname PY@tok@nl\endcsname{\def\PY@tc##1{\textcolor[rgb]{0.63,0.63,0.00}{##1}}}
\expandafter\def\csname PY@tok@ni\endcsname{\let\PY@bf=\textbf\def\PY@tc##1{\textcolor[rgb]{0.60,0.60,0.60}{##1}}}
\expandafter\def\csname PY@tok@na\endcsname{\def\PY@tc##1{\textcolor[rgb]{0.49,0.56,0.16}{##1}}}
\expandafter\def\csname PY@tok@nt\endcsname{\let\PY@bf=\textbf\def\PY@tc##1{\textcolor[rgb]{0.00,0.50,0.00}{##1}}}
\expandafter\def\csname PY@tok@nd\endcsname{\def\PY@tc##1{\textcolor[rgb]{0.67,0.13,1.00}{##1}}}
\expandafter\def\csname PY@tok@s\endcsname{\def\PY@tc##1{\textcolor[rgb]{0.73,0.13,0.13}{##1}}}
\expandafter\def\csname PY@tok@sd\endcsname{\let\PY@it=\textit\def\PY@tc##1{\textcolor[rgb]{0.73,0.13,0.13}{##1}}}
\expandafter\def\csname PY@tok@si\endcsname{\let\PY@bf=\textbf\def\PY@tc##1{\textcolor[rgb]{0.73,0.40,0.53}{##1}}}
\expandafter\def\csname PY@tok@se\endcsname{\let\PY@bf=\textbf\def\PY@tc##1{\textcolor[rgb]{0.73,0.40,0.13}{##1}}}
\expandafter\def\csname PY@tok@sr\endcsname{\def\PY@tc##1{\textcolor[rgb]{0.73,0.40,0.53}{##1}}}
\expandafter\def\csname PY@tok@ss\endcsname{\def\PY@tc##1{\textcolor[rgb]{0.10,0.09,0.49}{##1}}}
\expandafter\def\csname PY@tok@sx\endcsname{\def\PY@tc##1{\textcolor[rgb]{0.00,0.50,0.00}{##1}}}
\expandafter\def\csname PY@tok@m\endcsname{\def\PY@tc##1{\textcolor[rgb]{0.40,0.40,0.40}{##1}}}
\expandafter\def\csname PY@tok@gh\endcsname{\let\PY@bf=\textbf\def\PY@tc##1{\textcolor[rgb]{0.00,0.00,0.50}{##1}}}
\expandafter\def\csname PY@tok@gu\endcsname{\let\PY@bf=\textbf\def\PY@tc##1{\textcolor[rgb]{0.50,0.00,0.50}{##1}}}
\expandafter\def\csname PY@tok@gd\endcsname{\def\PY@tc##1{\textcolor[rgb]{0.63,0.00,0.00}{##1}}}
\expandafter\def\csname PY@tok@gi\endcsname{\def\PY@tc##1{\textcolor[rgb]{0.00,0.63,0.00}{##1}}}
\expandafter\def\csname PY@tok@gr\endcsname{\def\PY@tc##1{\textcolor[rgb]{1.00,0.00,0.00}{##1}}}
\expandafter\def\csname PY@tok@ge\endcsname{\let\PY@it=\textit}
\expandafter\def\csname PY@tok@gs\endcsname{\let\PY@bf=\textbf}
\expandafter\def\csname PY@tok@gp\endcsname{\let\PY@bf=\textbf\def\PY@tc##1{\textcolor[rgb]{0.00,0.00,0.50}{##1}}}
\expandafter\def\csname PY@tok@go\endcsname{\def\PY@tc##1{\textcolor[rgb]{0.53,0.53,0.53}{##1}}}
\expandafter\def\csname PY@tok@gt\endcsname{\def\PY@tc##1{\textcolor[rgb]{0.00,0.27,0.87}{##1}}}
\expandafter\def\csname PY@tok@err\endcsname{\def\PY@bc##1{\setlength{\fboxsep}{0pt}\fcolorbox[rgb]{1.00,0.00,0.00}{1,1,1}{\strut ##1}}}
\expandafter\def\csname PY@tok@kc\endcsname{\let\PY@bf=\textbf\def\PY@tc##1{\textcolor[rgb]{0.00,0.50,0.00}{##1}}}
\expandafter\def\csname PY@tok@kd\endcsname{\let\PY@bf=\textbf\def\PY@tc##1{\textcolor[rgb]{0.00,0.50,0.00}{##1}}}
\expandafter\def\csname PY@tok@kn\endcsname{\let\PY@bf=\textbf\def\PY@tc##1{\textcolor[rgb]{0.00,0.50,0.00}{##1}}}
\expandafter\def\csname PY@tok@kr\endcsname{\let\PY@bf=\textbf\def\PY@tc##1{\textcolor[rgb]{0.00,0.50,0.00}{##1}}}
\expandafter\def\csname PY@tok@bp\endcsname{\def\PY@tc##1{\textcolor[rgb]{0.00,0.50,0.00}{##1}}}
\expandafter\def\csname PY@tok@fm\endcsname{\def\PY@tc##1{\textcolor[rgb]{0.00,0.00,1.00}{##1}}}
\expandafter\def\csname PY@tok@vc\endcsname{\def\PY@tc##1{\textcolor[rgb]{0.10,0.09,0.49}{##1}}}
\expandafter\def\csname PY@tok@vg\endcsname{\def\PY@tc##1{\textcolor[rgb]{0.10,0.09,0.49}{##1}}}
\expandafter\def\csname PY@tok@vi\endcsname{\def\PY@tc##1{\textcolor[rgb]{0.10,0.09,0.49}{##1}}}
\expandafter\def\csname PY@tok@vm\endcsname{\def\PY@tc##1{\textcolor[rgb]{0.10,0.09,0.49}{##1}}}
\expandafter\def\csname PY@tok@sa\endcsname{\def\PY@tc##1{\textcolor[rgb]{0.73,0.13,0.13}{##1}}}
\expandafter\def\csname PY@tok@sb\endcsname{\def\PY@tc##1{\textcolor[rgb]{0.73,0.13,0.13}{##1}}}
\expandafter\def\csname PY@tok@sc\endcsname{\def\PY@tc##1{\textcolor[rgb]{0.73,0.13,0.13}{##1}}}
\expandafter\def\csname PY@tok@dl\endcsname{\def\PY@tc##1{\textcolor[rgb]{0.73,0.13,0.13}{##1}}}
\expandafter\def\csname PY@tok@s2\endcsname{\def\PY@tc##1{\textcolor[rgb]{0.73,0.13,0.13}{##1}}}
\expandafter\def\csname PY@tok@sh\endcsname{\def\PY@tc##1{\textcolor[rgb]{0.73,0.13,0.13}{##1}}}
\expandafter\def\csname PY@tok@s1\endcsname{\def\PY@tc##1{\textcolor[rgb]{0.73,0.13,0.13}{##1}}}
\expandafter\def\csname PY@tok@mb\endcsname{\def\PY@tc##1{\textcolor[rgb]{0.40,0.40,0.40}{##1}}}
\expandafter\def\csname PY@tok@mf\endcsname{\def\PY@tc##1{\textcolor[rgb]{0.40,0.40,0.40}{##1}}}
\expandafter\def\csname PY@tok@mh\endcsname{\def\PY@tc##1{\textcolor[rgb]{0.40,0.40,0.40}{##1}}}
\expandafter\def\csname PY@tok@mi\endcsname{\def\PY@tc##1{\textcolor[rgb]{0.40,0.40,0.40}{##1}}}
\expandafter\def\csname PY@tok@il\endcsname{\def\PY@tc##1{\textcolor[rgb]{0.40,0.40,0.40}{##1}}}
\expandafter\def\csname PY@tok@mo\endcsname{\def\PY@tc##1{\textcolor[rgb]{0.40,0.40,0.40}{##1}}}
\expandafter\def\csname PY@tok@ch\endcsname{\let\PY@it=\textit\def\PY@tc##1{\textcolor[rgb]{0.25,0.50,0.50}{##1}}}
\expandafter\def\csname PY@tok@cm\endcsname{\let\PY@it=\textit\def\PY@tc##1{\textcolor[rgb]{0.25,0.50,0.50}{##1}}}
\expandafter\def\csname PY@tok@cpf\endcsname{\let\PY@it=\textit\def\PY@tc##1{\textcolor[rgb]{0.25,0.50,0.50}{##1}}}
\expandafter\def\csname PY@tok@c1\endcsname{\let\PY@it=\textit\def\PY@tc##1{\textcolor[rgb]{0.25,0.50,0.50}{##1}}}
\expandafter\def\csname PY@tok@cs\endcsname{\let\PY@it=\textit\def\PY@tc##1{\textcolor[rgb]{0.25,0.50,0.50}{##1}}}

\def\PYZbs{\char`\\}
\def\PYZus{\char`\_}
\def\PYZob{\char`\{}
\def\PYZcb{\char`\}}
\def\PYZca{\char`\^}
\def\PYZam{\char`\&}
\def\PYZlt{\char`\<}
\def\PYZgt{\char`\>}
\def\PYZsh{\char`\#}
\def\PYZpc{\char`\%}
\def\PYZdl{\char`\$}
\def\PYZhy{\char`\-}
\def\PYZsq{\char`\'}
\def\PYZdq{\char`\"}
\def\PYZti{\char`\~}
% for compatibility with earlier versions
\def\PYZat{@}
\def\PYZlb{[}
\def\PYZrb{]}
\makeatother

% ANSI colors
\definecolor{ansi-black}{HTML}{3E424D}
\definecolor{ansi-black-intense}{HTML}{282C36}
\definecolor{ansi-red}{HTML}{E75C58}
\definecolor{ansi-red-intense}{HTML}{B22B31}
\definecolor{ansi-green}{HTML}{00A250}
\definecolor{ansi-green-intense}{HTML}{007427}
\definecolor{ansi-yellow}{HTML}{DDB62B}
\definecolor{ansi-yellow-intense}{HTML}{B27D12}
\definecolor{ansi-blue}{HTML}{208FFB}
\definecolor{ansi-blue-intense}{HTML}{0065CA}
\definecolor{ansi-magenta}{HTML}{D160C4}
\definecolor{ansi-magenta-intense}{HTML}{A03196}
\definecolor{ansi-cyan}{HTML}{60C6C8}
\definecolor{ansi-cyan-intense}{HTML}{258F8F}
\definecolor{ansi-white}{HTML}{C5C1B4}
\definecolor{ansi-white-intense}{HTML}{A1A6B2}

% commands and environments needed by pandoc snippets
% extracted from the output of `pandoc -s`
\providecommand{\tightlist}{%
  \setlength{\itemsep}{0pt}\setlength{\parskip}{0pt}}
\DefineVerbatimEnvironment{Highlighting}{Verbatim}{commandchars=\\\{\}}
% Add ',fontsize=\small' for more characters per line
\newenvironment{Shaded}{}{}
\newcommand{\KeywordTok}[1]{\textcolor[rgb]{0.00,0.44,0.13}{\textbf{{#1}}}}
\newcommand{\DataTypeTok}[1]{\textcolor[rgb]{0.56,0.13,0.00}{{#1}}}
\newcommand{\DecValTok}[1]{\textcolor[rgb]{0.25,0.63,0.44}{{#1}}}
\newcommand{\BaseNTok}[1]{\textcolor[rgb]{0.25,0.63,0.44}{{#1}}}
\newcommand{\FloatTok}[1]{\textcolor[rgb]{0.25,0.63,0.44}{{#1}}}
\newcommand{\CharTok}[1]{\textcolor[rgb]{0.25,0.44,0.63}{{#1}}}
\newcommand{\StringTok}[1]{\textcolor[rgb]{0.25,0.44,0.63}{{#1}}}
\newcommand{\CommentTok}[1]{\textcolor[rgb]{0.38,0.63,0.69}{\textit{{#1}}}}
\newcommand{\OtherTok}[1]{\textcolor[rgb]{0.00,0.44,0.13}{{#1}}}
\newcommand{\AlertTok}[1]{\textcolor[rgb]{1.00,0.00,0.00}{\textbf{{#1}}}}
\newcommand{\FunctionTok}[1]{\textcolor[rgb]{0.02,0.16,0.49}{{#1}}}
\newcommand{\RegionMarkerTok}[1]{{#1}}
\newcommand{\ErrorTok}[1]{\textcolor[rgb]{1.00,0.00,0.00}{\textbf{{#1}}}}
\newcommand{\NormalTok}[1]{{#1}}

% Additional commands for more recent versions of Pandoc
\newcommand{\ConstantTok}[1]{\textcolor[rgb]{0.53,0.00,0.00}{{#1}}}
\newcommand{\SpecialCharTok}[1]{\textcolor[rgb]{0.25,0.44,0.63}{{#1}}}
\newcommand{\VerbatimStringTok}[1]{\textcolor[rgb]{0.25,0.44,0.63}{{#1}}}
\newcommand{\SpecialStringTok}[1]{\textcolor[rgb]{0.73,0.40,0.53}{{#1}}}
\newcommand{\ImportTok}[1]{{#1}}
\newcommand{\DocumentationTok}[1]{\textcolor[rgb]{0.73,0.13,0.13}{\textit{{#1}}}}
\newcommand{\AnnotationTok}[1]{\textcolor[rgb]{0.38,0.63,0.69}{\textbf{\textit{{#1}}}}}
\newcommand{\CommentVarTok}[1]{\textcolor[rgb]{0.38,0.63,0.69}{\textbf{\textit{{#1}}}}}
\newcommand{\VariableTok}[1]{\textcolor[rgb]{0.10,0.09,0.49}{{#1}}}
\newcommand{\ControlFlowTok}[1]{\textcolor[rgb]{0.00,0.44,0.13}{\textbf{{#1}}}}
\newcommand{\OperatorTok}[1]{\textcolor[rgb]{0.40,0.40,0.40}{{#1}}}
\newcommand{\BuiltInTok}[1]{{#1}}
\newcommand{\ExtensionTok}[1]{{#1}}
\newcommand{\PreprocessorTok}[1]{\textcolor[rgb]{0.74,0.48,0.00}{{#1}}}
\newcommand{\AttributeTok}[1]{\textcolor[rgb]{0.49,0.56,0.16}{{#1}}}
\newcommand{\InformationTok}[1]{\textcolor[rgb]{0.38,0.63,0.69}{\textbf{\textit{{#1}}}}}
\newcommand{\WarningTok}[1]{\textcolor[rgb]{0.38,0.63,0.69}{\textbf{\textit{{#1}}}}}

% Define a nice break command that doesn't care if a line doesn't already
% exist.
\def\br{\hspace*{\fill} \\* }

% Math Jax compatability definitions
\def\gt{>}
\def\lt{<}

\setcounter{secnumdepth}{5}

% Colors for the hyperref package
\definecolor{urlcolor}{rgb}{0,.145,.698}
\definecolor{linkcolor}{rgb}{.71,0.21,0.01}
\definecolor{citecolor}{rgb}{.12,.54,.11}

\DeclareTranslationFallback{Author}{Author}
\DeclareTranslation{Portuges}{Author}{Autor}

\DeclareTranslationFallback{List of Codes}{List of Codes}
\DeclareTranslation{Catalan}{List of Codes}{Llista de Codis}
\DeclareTranslation{Danish}{List of Codes}{Liste over Koder}
\DeclareTranslation{German}{List of Codes}{Liste der Codes}
\DeclareTranslation{Spanish}{List of Codes}{Lista de C\'{o}digos}
\DeclareTranslation{French}{List of Codes}{Liste des Codes}
\DeclareTranslation{Italian}{List of Codes}{Elenco dei Codici}
\DeclareTranslation{Dutch}{List of Codes}{Lijst van Codes}
\DeclareTranslation{Portuges}{List of Codes}{Lista de C\'{o}digos}

\DeclareTranslationFallback{Supervisors}{Supervisors}
\DeclareTranslation{Catalan}{Supervisors}{Supervisors}
\DeclareTranslation{Danish}{Supervisors}{Vejledere}
\DeclareTranslation{German}{Supervisors}{Vorgesetzten}
\DeclareTranslation{Spanish}{Supervisors}{Supervisores}
\DeclareTranslation{French}{Supervisors}{Superviseurs}
\DeclareTranslation{Italian}{Supervisors}{Le autorit\`{a} di vigilanza}
\DeclareTranslation{Dutch}{Supervisors}{supervisors}
\DeclareTranslation{Portuguese}{Supervisors}{Supervisores}

\definecolor{codegreen}{rgb}{0,0.6,0}
\definecolor{codegray}{rgb}{0.5,0.5,0.5}
\definecolor{codepurple}{rgb}{0.58,0,0.82}
\definecolor{backcolour}{rgb}{0.95,0.95,0.95}

\lstdefinestyle{mystyle}{
    commentstyle=\color{codegreen},
    keywordstyle=\color{magenta},
    numberstyle=\tiny\color{codegray},
    stringstyle=\color{codepurple},
    basicstyle=\ttfamily,
    breakatwhitespace=false,
    keepspaces=true,
    numbers=left,
    numbersep=10pt,
    showspaces=false,
    showstringspaces=false,
    showtabs=false,
    tabsize=2,
    breaklines=true,
    literate={\-}{}{0\discretionary{-}{}{-}},
  postbreak=\mbox{\textcolor{red}{$\hookrightarrow$}\space},
}

\lstset{style=mystyle}

\surroundwithmdframed[
  hidealllines=true,
  backgroundcolor=backcolour,
  innerleftmargin=0pt,
  innerrightmargin=0pt,
  innertopmargin=0pt,
  innerbottommargin=0pt]{lstlisting}

%%%%%%%%%%%%

%%%%%%%%%%%% MARGINS

 % Used to adjust the document margins
\usepackage{geometry}
\geometry{tmargin=1in,bmargin=1in,lmargin=1in,rmargin=1in,
nohead,includefoot,footskip=25pt}
% you can use showframe option to check the margins visually
%%%%%%%%%%%%

%%%%%%%%%%%% COMMANDS

% ensure new section starts on new page
\addtokomafont{section}{\clearpage}

% Prevent overflowing lines due to hard-to-break entities
\sloppy

% Setup hyperref package
\hypersetup{
    breaklinks=true,  % so long urls are correctly broken across lines
    colorlinks=true,
    urlcolor=urlcolor,
    linkcolor=linkcolor,
    citecolor=citecolor,
    }

% ensure figures are placed within subsections
\makeatletter
\AtBeginDocument{%
    \expandafter\renewcommand\expandafter\subsection\expandafter
    {\expandafter\@fb@secFB\subsection}%
    \newcommand\@fb@secFB{\FloatBarrier
    \gdef\@fb@afterHHook{\@fb@topbarrier \gdef\@fb@afterHHook{}}}%
    \g@addto@macro\@afterheading{\@fb@afterHHook}%
    \gdef\@fb@afterHHook{}%
}
\makeatother

% number figures, tables and equations by section
\counterwithout{figure}{section}
\counterwithout{table}{section}
\counterwithout{equation}{section}
\makeatletter
\@addtoreset{table}{section}
\@addtoreset{figure}{section}
\@addtoreset{equation}{section}
\makeatother
\renewcommand\thetable{\thesection.\arabic{table}}
\renewcommand\thefigure{\thesection.\arabic{figure}}
\renewcommand\theequation{\thesection.\arabic{equation}}

    % set global options for float placement
    \makeatletter
        \providecommand*\setfloatlocations[2]{\@namedef{fps@#1}{#2}}
    \makeatother

% align captions to left (indented)
\captionsetup{justification=raggedright,
singlelinecheck=false,format=hang,labelfont={it,bf}}

% shift footer down so space between separation line
\ModifyLayer[addvoffset=.6ex]{scrheadings.foot.odd}
\ModifyLayer[addvoffset=.6ex]{scrheadings.foot.even}
\ModifyLayer[addvoffset=.6ex]{scrheadings.foot.oneside}
\ModifyLayer[addvoffset=.6ex]{plain.scrheadings.foot.odd}
\ModifyLayer[addvoffset=.6ex]{plain.scrheadings.foot.even}
\ModifyLayer[addvoffset=.6ex]{plain.scrheadings.foot.oneside}
\pagestyle{scrheadings}
\clearscrheadfoot{}
\ifoot{\leftmark}
\renewcommand{\sectionmark}[1]{\markleft{\thesection\ #1}}
\ofoot{\pagemark}
\cfoot{}

%%%%%%%%%%%%

%%%%%%%%%%%% FINAL HEADER MATERIAL

% clereref must be loaded after anything that changes the referencing system
\usepackage{cleveref}
\creflabelformat{equation}{#2#1#3}

% make the code float work with cleverref
\crefname{codecell}{code}{codes}
\Crefname{codecell}{code}{codes}
% make the text float work with cleverref
\crefname{textcell}{text}{texts}
\Crefname{textcell}{text}{texts}
% make the text float work with cleverref
\crefname{errorcell}{error}{errors}
\Crefname{errorcell}{error}{errors}

%%%%%%%%%%%%

\begin{document}

    \begin{titlepage}

  \begin{center}

  \vspace*{1cm}

  \Huge\textbf{Overpressure Analysis of Girard Point Refinery Accident}

  \vspace{0.5cm}\LARGE{Boiling-Liquid Expanding-Vapor Explosion}

  \vspace{1.5cm}

  \begin{minipage}{0.8\textwidth}
    \begin{center}
    \begin{minipage}{0.39\textwidth}
    \begin{flushleft} \Large
    \emph{\GetTranslation{Author}:}\\S. Kevin McNeill\\\href{mailto:shonn.mcneill@atf.gov}{shonn.mcneill@atf.gov}
    \end{flushleft}
    \end{minipage}
    \hspace{\fill}
    \begin{minipage}{0.39\textwidth}
    \begin{flushright} \Large
    \end{flushright}
    \end{minipage}
    \end{center}
  \end{minipage}

  \vfill

  \begin{minipage}{0.8\textwidth}
  \begin{center}
  \end{center}
  \end{minipage}

  \vspace{0.8cm}
      \LARGE{National Center for Explosives Training and Research}\\
      \LARGE{Explosives Research and Development Division}\\

  \vspace{0.4cm}

  \today

  \end{center}
  \end{titlepage}

    \begingroup
    \let\cleardoublepage\relax
    \let\clearpage\relax\tableofcontents\listoffigures\listoftables\listof{codecell}{\GetTranslation{List of Codes}}
    \endgroup

\hypertarget{introduction}{%
\section{Introduction}\label{introduction}}

On June 21, 2019, an accidental fire and explosion occurred at the
Girard Point Refinery owned by Philadelphia Energy
Solutions\href{https://6abc.com/hydrocarbon-vapors-identified-as-cause-of-philly-refinery-fire/5368469/}{a}.
One of the three explosions observed was a storage tank weighing 74,000
lb containing 90,000 lb of iso-butane, propane, and hydrofluoric acid.
The tank was propelled approximately 2,000 ft from the blast seat. The
Philadelphia Fire Department requested ATF estimate the blast
overpressure generated when the tank exploded.

It is hypothesized that a boiling-liquid expanding-vapor explosion
(BLEVE) event provided the energy to generate a blast wave. This
notebook is an engineering analysis to estimate the blast overpressure.
The analysis is based upon an adiabatic but irreversible energy analysis
developed by Planas-Cuchi et.al {[}1{]}.

\hypertarget{background}{%
\section{Background}\label{background}}

This analysis calculates the pressure of the blastwave generated from
the BLEVE by finding the equivalent mass of TNT that would release the
same amount of energy. With the mass of TNT known, a pressure can be
calculated using the Kingery-Bulmash equations. The energy released from
the BLEVE is calculated using the 1st Law of Thermodynamics assuming an
adiabatic irreversible process.\\
\#\# Boiling-Liquid Expanding-Vapor Explosion A BLEVE results from the
sudden failure of a tank containing a compressed vapor (head space) and
a super-heated liquid (a liquid heated above it's boiling point but
without boiling). The magnitude of the blast depends on how super-heated
the liquid was at failure. As the level of super-heat rises, the portion
of liquid that flash-boils rises, thus increasing the energy released.
Once containment failure occurs the energy is distributed into four
forms:

\begin{enumerate}
\def\labelenumi{\arabic{enumi}.}
\tightlist
\item
  Overpressure wave
\item
  Kinetic energy of fragments
\item
  Deformation and failure of the containment material
\item
  Heat transferred to environment
\end{enumerate}

The distribution of the energy into the these four forms depends on the
specifics of the explosion. Casal et al.~found that a \emph{fragile}
failure releases 80\% of the energy into the blastwave, while a
\emph{ductile} failure releases 40\% of the energy into the blastwave.
The remaining energy becomes kinetic energy of the fragments. The heat
transfer to the environment is relatively small.{[}2{]} In practice most
pressure vessels are designed with materials that are ductile rather
than brittle to avoid sudden and catastrophic brittle (fragile) failures
{[}3{]}.

\hypertarget{first-law-of-thermodynamics}{%
\subsection{First Law of
Thermodynamics}\label{first-law-of-thermodynamics}}

The 1st Law of Thermodynamics is a restatement of the conservation of
energy (energy cannot be created or destroyed). A way of stating the
first law of thermodynamics is the internal energy \((\Delta U)\) of a
system is given by the sum of the heat \((Q)\) that flows into and
out-of the system and the work \((W)\) done on the system:

\begin{equation}\Delta U = Q+W\end{equation}

Therefore, there are only two ways to change the internal energy
\(\Delta U\) of the system:

\begin{itemize}
\tightlist
\item
  Add or remove heat from the system, \(Q\).
\item
  Perform or extract work from the system, \(W\).
\end{itemize}

It is assumed for this analysis the BLEVE expansion of the super-heated
liquid (initial state) to atmospheric conditions (final state) occurs so
quickly there is no time for heat to be added or removed from the
system. When heat cannot be added or removed from a system it is called
an adiabatic process. During an adiabatic process the heat \((Q)\) is
zero and the 1st Law of Thermodynamics can be rewritten to:

\begin{equation}\Delta U = W\end{equation} \#\# Work The work \(W\) of
an expanding gas is given by,

\begin{equation}W = P_0\Delta V\end{equation}

Where \(P_0\) is the initial pressure and \(\Delta V\) is the difference
in the final and initial volumes. In the case of the BLEVE the change in
volume is due to the conversion of the super-heated liquid to a gas. We
know the initial volume of the system, \(V_i\) in this case, the tank
volume. However, the final volume, \(V_f\), is an unknown. We can
however transform the final volume into terms we do know. If we rewrite
the final volume \((V_f)\) in terms of the specific volume (volume
divided by mass) times the mass we can transform our change in final
volume into a change in final mass,

\begin{equation}\Delta V =  V_f - V_i\end{equation}
\begin{equation}\Delta V =  \left(\nu_Gm_G + \nu_Lm_L \right)_f - V_i\end{equation}

Therefore,

\begin{equation}W = P_0\left[\left(\nu_Gm_G + \nu_Lm_L \right)_f - V_i\right]\end{equation}

The quality of a gas-liquid mixture is defined as the fraction of the
total mass (gas + liquid) that is saturated vapor,

\begin{equation}\chi = \frac{m_G}{m_T}\end{equation}

Where \(m_G\) is the mass of the gas and \(m_T\) is the total mass.
Solving for the mass of the gas,

\begin{equation}m_G = m_T \chi\end{equation}

The mass of the liquid is then given by,

\begin{equation}m_T = m_G + m_L\end{equation}
\begin{equation}m_L = m_T - m_G\end{equation}
\begin{equation}m_L = m_T - m_T \chi\end{equation}

Substituting equations (7) and (10) back into (5) we have,

\begin{equation}W = P_0\left[\left( \nu_G m_T \chi + \nu_L(m_T - m_T \chi) \right)_f - V_i\right]\end{equation}
\begin{equation}W = P_0\left[\left( \nu_G m_T \chi + \nu_L m_T - \nu_L m_T \chi \right)_f - V_i\right]\end{equation}
\begin{equation}W = P_0\left[\left( \left(\nu_G - \nu_L \right) m_T \chi + \nu_L m_T \right)_f - V_i\right]\end{equation}

\hypertarget{internal-energy}{%
\subsection{Internal Energy}\label{internal-energy}}

The internal energy is the energy of atomic motion. It is the sum of the
molecular kinetic and potential energies in a system (in our case, the
tank filled with fluid). The molecular kinetic energy is made up of
three types of molecular motion:

\begin{itemize}
\tightlist
\item
  translational,
\item
  rotational, and
\item
  vibrational.
\end{itemize}

The molecular potential energy is derived from the electromagnetic force
that exists between the atoms in individual molecules and the
intermolecular forces between molecules. The internal energy is directly
proportional to the temperature of the system (higher temperatures mean
more internal energy and lower temperatures mean lower internal energy).

The change in internal energy can be changed similarly to how we change
the volume in the Work section. The change in internal energy can be
rewritten in terms of the specific internal energy (internal energy
divided by the mass) and the quality,

\begin{equation}\Delta U = U_f - U_i\end{equation}

\begin{equation}\Delta U = \left(u_Gm_G + u_Lm_L\right)_f - U_i\end{equation}

Substituting equations (7) and (10) into the internal energy equation
(15) we have,

\begin{equation}\Delta U = \left(u_Gm_T\chi + u_L(m_T - m_T \chi)\right)_f - U_i\end{equation}

\begin{equation}\Delta U = \left( u_G m_T \chi + u_L m_T - u_L m_T \chi \right)_f - U_i\end{equation}

\begin{equation}\Delta U = \left( \left(u_G - u_L \right) m_T \chi + u_L m_T \right)_f - U_i\end{equation}

Referring back to equation (3) and substituting equations (14) and (19)
we can solve for the quality \(\chi\),

\begin{equation}\chi = \frac{m_TP_0\nu_L-V_iP_0+m_Tu_L-U_i}{\left[(u_L-u_G)-(\nu_G-\nu_L)P_0 \right]m_T}\end{equation}

By substituting \(\chi\) back into equation (19) we can solve for the
internal energy, \(\Delta U\). Once we know the change in internal
energy due to the flash-boiling we can calculate the equivalent weight
of TNT that will produce the same change in internal energy.

\hypertarget{conversion-to-tnt-equivalent-weight}{%
\subsection{Conversion to TNT Equivalent
Weight}\label{conversion-to-tnt-equivalent-weight}}

The equation for the conversion to equivalent TNT weight in kilograms
is,

\begin{equation}W_{TNT} = \beta\: 0.2136 \Delta U\end{equation}

Where: - \(\beta\) is the fraction of the energy released converted to
the blast wave (fragile = 80\% and ductile = 40\%) -
\(0.2136\frac{kg}{MJ}\) is the inverse of blast energy of TNT
\(\left(4680\:J/g_{TNT}\right)\)\href{https://books.google.com/books?id=Id15BgAAQBAJ\&pg=PA2\&lpg=PA2\&dq=tnt+4680+J/g\&source=bl\&ots=qx-7s6EAHa\&sig=ACfU3U1wcca5cWLX38lBoxDvvRqH0uaZ5g\&hl=en\&sa=X\&ved=2ahUKEwiN2cDP1pHjAhWnm-AKHfBYCGkQ6AEwAHoECAkQAQ\#v=onepage\&q=tnt\%204680\%20J\%2Fg\&f=false}{4}
- \(\Delta U\) is the internal energy of the flash boiling liquid in
\(bar\:m^3\).

With the weight of TNT known it is a simple matter to calculate the
overpressure at a specific distance from the BLEVE using the Kingery
Bulmash equations {[}5{]}.

\hypertarget{notebook-imports}{%
\section{Notebook Imports}\label{notebook-imports}}

\begin{lstlisting}[language=Python,numbers=left,xleftmargin=20pt,xrightmargin=5pt,belowskip=5pt,aboveskip=5pt]

# Data manipulation
import pandas as pd
import numpy as np

# Options for pandas
pd.options.display.max_columns = 50
pd.options.display.max_rows = 30

# Display all cell outputs
from IPython.core.interactiveshell import InteractiveShell
InteractiveShell.ast_node_interactivity = 'all'

from IPython import get_ipython
ipython = get_ipython()

# autoreload extension
if 'autoreload' not in ipython.extension_manager.loaded:
    %load_ext autoreload

%autoreload 2

# Visualizations
import plotly as py
import plotly.graph_objs as go
import plotly.figure_factory as ff
from plotly.offline import iplot, init_notebook_mode
init_notebook_mode(connected=True)

import cufflinks as cf
cf.go_offline(connected=True)
cf.set_config_file(theme='white')

# Units and Thermodynamic Data
import pint as unit
import pyromat as pm
\end{lstlisting}

\begin{lstlisting}[language=Python,numbers=left,xleftmargin=20pt,xrightmargin=5pt,belowskip=5pt,aboveskip=5pt]
# Define a dictionary containing the study data
verify_data = {'parameter': [
                   'Pressure [kPa]', 'Temperature [C]',
                   'Total mass [kg]', 'Mass of liquid [kg]',
                   'Mass of vapour [kg]', 'Vapour specific volume [m3 kg-1]',
                   'Liquid specific volume [m3 kg-1]','System specific volume[m3 kg-1]',
                   'Total vapour volume [m3]', 'Total liquid volume [m3]',
                   'Total volume [m3]', 'Vapour fraction',
                   'Vapour specific internal energy [kJ kg-1]', 'Liquid specific internal energy [kJ kg-1]',
                   'System specific internal energy [kJ kg-1]', 'Total vapour internal energy [MJ]',
                   'Total liquid internal energy [MJ]', 'Total internal energy [MJ]',
                   'Vapour specific entropy [kJ kg-1 K-1]', 'Liquid specific entropy [kJ kg-1 K-1]',
                   'System specific entropy [kJ kg-1 K-1]'],
                   'explosion_state': [1901, 55, 100956, 100007, 949.1,0.02293, 0.002282, 0.002476, 21.8, 228.2, 250, 0.009401, 582, 349.2, 351.4, 560, 34920, 35480, 2.33, 1.501, 1.508],
                   'final_state': [101.3, -42.02, 100956, 41288, 59668, 0.4136, 0.001721, 0.2452, 24681, 71, 24752, 0.591, 483.7, 100.1, 326.8, 28860, 4133, 32990, 2.448, 0.6068, 1.695]}
# Convert the dictionary into DataFrame
df = pd.DataFrame(verify_data)
df.set_index('parameter');

\end{lstlisting}

\begin{lstlisting}[language=Python,numbers=left,xleftmargin=20pt,xrightmargin=5pt,belowskip=5pt,aboveskip=5pt]
def delta_U(uL, uG, mT, x, Ui):
    '''
    This function calculates the change in internal energy
    of an adiabatically expanding liquid.  Where u is the
    specific internal energy, L is liquid, G is gas,
    mT is total mass, x is thermodynamic quality, and Ui
    is the internal energy at the initial state.
    '''
    return (uL-uG)*mT*x-mT*uL+Ui

def P_delta_V(P0, vL, vG, mT, x, Vi):
    '''
    This function calculates the P(Vf-Vi)
    of an adiabatically expanding liquid.  Where v is the
    specific volume, L is liquid, G is gas,
    mT is total mass, x is thermodynamic quality,
    P0 is the atmospheric pressure,and Vi
    is the volume at the initial state.
    '''
    return P0*((vG-vL)*mT*x+mT*vL-Vi)

def X(uL, uG, mT, x, Ui, P0, vL, vG, Vi):
    '''
    This function calculates the quality
    of an adiabatically expanding liquid.  Where u is the
    specific internal energy, v is the specific volume,
    L is liquid, G is gas, mT is total mass,
    x is thermodynamic quality, P0 is the atmospheric and
    Ui is the internal energy at the initial state.
    '''
    return (mT*P0*vL-Vi*P0+mT*uL-Ui)/((uL-uG)-(vG-vL)*P0)*mT
\end{lstlisting}

\begin{lstlisting}[language=Python,numbers=left,xleftmargin=20pt,xrightmargin=5pt,belowskip=5pt,aboveskip=5pt]
df.loc['Pressure [kPa]']['explosion_state']
\end{lstlisting}

\begin{lstlisting}[language=Python,numbers=none,xrightmargin=5pt,belowskip=2pt,aboveskip=2pt]
    ---------------------------------------------------------------------------
\end{lstlisting}

\begin{lstlisting}[language=Python,numbers=none,xrightmargin=5pt,belowskip=2pt,aboveskip=2pt]
    TypeError                                 Traceback (most recent call last)
\end{lstlisting}

\begin{lstlisting}[language=Python,numbers=none,xrightmargin=5pt,belowskip=2pt,aboveskip=2pt]
    <ipython-input-9-5db034eab431> in <module>
    ----> 1 df.iloc['Pressure [kPa]']['explosion_state']

\end{lstlisting}

\begin{lstlisting}[language=Python,numbers=none,xrightmargin=5pt,belowskip=2pt,aboveskip=2pt]
    ~/miniconda3/envs/philadelphia_refinery/lib/python3.7/site-packages/pandas/core/indexing.py in __getitem__(self, key)
       1498
       1499             maybe_callable = com.apply_if_callable(key, self.obj)
    -> 1500             return self._getitem_axis(maybe_callable, axis=axis)
       1501
       1502     def _is_scalar_access(self, key):

\end{lstlisting}

\begin{lstlisting}[language=Python,numbers=none,xrightmargin=5pt,belowskip=2pt,aboveskip=2pt]
    ~/miniconda3/envs/philadelphia_refinery/lib/python3.7/site-packages/pandas/core/indexing.py in _getitem_axis(self, key, axis)
       2224         else:
       2225             if not is_integer(key):
    -> 2226                 raise TypeError("Cannot index by location index with a "
       2227                                 "non-integer key")
       2228

\end{lstlisting}

\begin{lstlisting}[language=Python,numbers=none,xrightmargin=5pt,belowskip=2pt,aboveskip=2pt]
    TypeError: Cannot index by location index with a non-integer key
\end{lstlisting}

\begin{lstlisting}[language=Python,numbers=left,xleftmargin=20pt,xrightmargin=5pt,belowskip=5pt,aboveskip=5pt]
propane = pm.get('ig.C3H8')

T = list(range(300, 4000))
data = [go.Scatter(x=T,y=propane.e(T))]
layout = go.Layout(xaxis={'title':'Temperature (K)'},
                   yaxis={'title':'Internal Energy, U (kJ/kg)'})
figure=go.Figure(data=data,layout=layout)
py.offline.iplot(figure)
\end{lstlisting}

\hypertarget{cleanup-data}{%
\section{Cleanup Data}\label{cleanup-data}}

Like families, tidy datasets are all alike but every messy dataset is
messy in its own way. Tidy datasets provide a standardized way to link
the structure of a dataset (its physical layout) with its semantics (its
meaning). In this section, I'll provide some standard vocabulary for
describing the structure and semantics of a dataset, and then use those
definitions to define tidy data.
\href{https://tidyr.tidyverse.org/articles/tidy-data.html}{tidyr}

\hypertarget{analysismodeling}{%
\section{Analysis/Modeling}\label{analysismodeling}}

\hypertarget{results}{%
\section{Results}\label{results}}

\begin{itemize}
\tightlist
\item
  Focus on main analysis steps and findings
\item
  Remove intermediate results or move to supplement
\end{itemize}

\hypertarget{conclusion}{%
\section{Conclusion}\label{conclusion}}

\hypertarget{references}{%
\section{References}\label{references}}

\begin{enumerate}
\def\labelenumi{\arabic{enumi}.}
\tightlist
\item
  Planas-Cuchi, E., Salla, J. M., \& Casal, J. (2004). Calculating
  overpressure from BLEVE explosions. Journal of Loss Prevention in the
  Process Industries, 17(6), 431--436.
  https://doi.org/10.1016/j.jlp.2004.08.002
\end{enumerate}

\end{document}
