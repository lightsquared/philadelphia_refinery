

% A latex document created by ipypublish
% outline: ipypublish.templates.outline_schemas/latex_outline.latex.j2
% with segments:
% - standard-standard_packages: with standard nbconvert packages
% - standard-standard_definitions: with standard nbconvert definitions
% - ipypublish-doc_article: with the main ipypublish article setup
% - ipypublish-front_pages: with the main ipypublish title and contents page setup
% - ipypublish-biblio_natbib: with the main ipypublish bibliography
% - ipypublish-contents_output: with the main ipypublish content
% - ipypublish-contents_framed_code: with the input code wrapped and framed
% - ipypublish-glossary: with the main ipypublish glossary
%
%%%%%%%%%%%% DOCCLASS

\documentclass[10pt,parskip=half,
toc=sectionentrywithdots,
bibliography=totocnumbered,
captions=tableheading,numbers=noendperiod]{scrartcl}

%%%%%%%%%%%%

%%%%%%%%%%%% PACKAGES

\usepackage[T1]{fontenc} % Nicer default font (+ math font) than Computer Modern for most use cases
\usepackage{mathpazo}
\usepackage{graphicx}
\usepackage[skip=3pt]{caption}
\usepackage{adjustbox} % Used to constrain images to a maximum size
\usepackage[table]{xcolor} % Allow colors to be defined
\usepackage{enumerate} % Needed for markdown enumerations to work
\usepackage{amsmath} % Equations
\usepackage{amssymb} % Equations
\usepackage{textcomp} % defines textquotesingle
% Hack from http://tex.stackexchange.com/a/47451/13684:
\AtBeginDocument{%
    \def\PYZsq{\textquotesingle}% Upright quotes in Pygmentized code
}
\usepackage{upquote} % Upright quotes for verbatim code
\usepackage{eurosym} % defines \euro
\usepackage[mathletters]{ucs} % Extended unicode (utf-8) support
\usepackage[utf8x]{inputenc} % Allow utf-8 characters in the tex document
\usepackage{fancyvrb} % verbatim replacement that allows latex
\usepackage{grffile} % extends the file name processing of package graphics
                        % to support a larger range
% The hyperref package gives us a pdf with properly built
% internal navigation ('pdf bookmarks' for the table of contents,
% internal cross-reference links, web links for URLs, etc.)
\usepackage{hyperref}
\usepackage{longtable} % longtable support required by pandoc >1.10
\usepackage{booktabs}  % table support for pandoc > 1.12.2
\usepackage[inline]{enumitem} % IRkernel/repr support (it uses the enumerate* environment)
\usepackage[normalem]{ulem} % ulem is needed to support strikethroughs (\sout)
                            % normalem makes italics be italics, not underlines

\usepackage{translations}
\usepackage{microtype} % improves the spacing between words and letters
\usepackage{placeins} % placement of figures
% could use \usepackage[section]{placeins} but placing in subsection in command section
% Places the float at precisely the location in the LaTeX code (with H)
\usepackage{float}
\usepackage[colorinlistoftodos,obeyFinal,textwidth=.8in]{todonotes} % to mark to-dos
% number figures, tables and equations by section
% fix for new versions of texlive (see https://tex.stackexchange.com/a/425603/107738)
\let\counterwithout\relax
\let\counterwithin\relax
\usepackage{chngcntr}
% header/footer
\usepackage[footsepline=0.25pt]{scrlayer-scrpage}

% bibliography formatting
\usepackage[sort,super]{natbib}
% hyperlink doi's
\usepackage{doi}

    % define a code float
    \usepackage{newfloat} % to define a new float types
    \DeclareFloatingEnvironment[
        fileext=frm,placement={!ht},
        within=section,name=Code]{codecell}
    \DeclareFloatingEnvironment[
        fileext=frm,placement={!ht},
        within=section,name=Text]{textcell}
    \DeclareFloatingEnvironment[
        fileext=frm,placement={!ht},
        within=section,name=Text]{errorcell}

    \usepackage{listings} % a package for wrapping code in a box
    \usepackage[framemethod=tikz]{mdframed} % to fram code

%%%%%%%%%%%%

%%%%%%%%%%%% DEFINITIONS

% Pygments definitions

\makeatletter
\def\PY@reset{\let\PY@it=\relax \let\PY@bf=\relax%
    \let\PY@ul=\relax \let\PY@tc=\relax%
    \let\PY@bc=\relax \let\PY@ff=\relax}
\def\PY@tok#1{\csname PY@tok@#1\endcsname}
\def\PY@toks#1+{\ifx\relax#1\empty\else%
    \PY@tok{#1}\expandafter\PY@toks\fi}
\def\PY@do#1{\PY@bc{\PY@tc{\PY@ul{%
    \PY@it{\PY@bf{\PY@ff{#1}}}}}}}
\def\PY#1#2{\PY@reset\PY@toks#1+\relax+\PY@do{#2}}

\expandafter\def\csname PY@tok@w\endcsname{\def\PY@tc##1{\textcolor[rgb]{0.73,0.73,0.73}{##1}}}
\expandafter\def\csname PY@tok@c\endcsname{\let\PY@it=\textit\def\PY@tc##1{\textcolor[rgb]{0.25,0.50,0.50}{##1}}}
\expandafter\def\csname PY@tok@cp\endcsname{\def\PY@tc##1{\textcolor[rgb]{0.74,0.48,0.00}{##1}}}
\expandafter\def\csname PY@tok@k\endcsname{\let\PY@bf=\textbf\def\PY@tc##1{\textcolor[rgb]{0.00,0.50,0.00}{##1}}}
\expandafter\def\csname PY@tok@kp\endcsname{\def\PY@tc##1{\textcolor[rgb]{0.00,0.50,0.00}{##1}}}
\expandafter\def\csname PY@tok@kt\endcsname{\def\PY@tc##1{\textcolor[rgb]{0.69,0.00,0.25}{##1}}}
\expandafter\def\csname PY@tok@o\endcsname{\def\PY@tc##1{\textcolor[rgb]{0.40,0.40,0.40}{##1}}}
\expandafter\def\csname PY@tok@ow\endcsname{\let\PY@bf=\textbf\def\PY@tc##1{\textcolor[rgb]{0.67,0.13,1.00}{##1}}}
\expandafter\def\csname PY@tok@nb\endcsname{\def\PY@tc##1{\textcolor[rgb]{0.00,0.50,0.00}{##1}}}
\expandafter\def\csname PY@tok@nf\endcsname{\def\PY@tc##1{\textcolor[rgb]{0.00,0.00,1.00}{##1}}}
\expandafter\def\csname PY@tok@nc\endcsname{\let\PY@bf=\textbf\def\PY@tc##1{\textcolor[rgb]{0.00,0.00,1.00}{##1}}}
\expandafter\def\csname PY@tok@nn\endcsname{\let\PY@bf=\textbf\def\PY@tc##1{\textcolor[rgb]{0.00,0.00,1.00}{##1}}}
\expandafter\def\csname PY@tok@ne\endcsname{\let\PY@bf=\textbf\def\PY@tc##1{\textcolor[rgb]{0.82,0.25,0.23}{##1}}}
\expandafter\def\csname PY@tok@nv\endcsname{\def\PY@tc##1{\textcolor[rgb]{0.10,0.09,0.49}{##1}}}
\expandafter\def\csname PY@tok@no\endcsname{\def\PY@tc##1{\textcolor[rgb]{0.53,0.00,0.00}{##1}}}
\expandafter\def\csname PY@tok@nl\endcsname{\def\PY@tc##1{\textcolor[rgb]{0.63,0.63,0.00}{##1}}}
\expandafter\def\csname PY@tok@ni\endcsname{\let\PY@bf=\textbf\def\PY@tc##1{\textcolor[rgb]{0.60,0.60,0.60}{##1}}}
\expandafter\def\csname PY@tok@na\endcsname{\def\PY@tc##1{\textcolor[rgb]{0.49,0.56,0.16}{##1}}}
\expandafter\def\csname PY@tok@nt\endcsname{\let\PY@bf=\textbf\def\PY@tc##1{\textcolor[rgb]{0.00,0.50,0.00}{##1}}}
\expandafter\def\csname PY@tok@nd\endcsname{\def\PY@tc##1{\textcolor[rgb]{0.67,0.13,1.00}{##1}}}
\expandafter\def\csname PY@tok@s\endcsname{\def\PY@tc##1{\textcolor[rgb]{0.73,0.13,0.13}{##1}}}
\expandafter\def\csname PY@tok@sd\endcsname{\let\PY@it=\textit\def\PY@tc##1{\textcolor[rgb]{0.73,0.13,0.13}{##1}}}
\expandafter\def\csname PY@tok@si\endcsname{\let\PY@bf=\textbf\def\PY@tc##1{\textcolor[rgb]{0.73,0.40,0.53}{##1}}}
\expandafter\def\csname PY@tok@se\endcsname{\let\PY@bf=\textbf\def\PY@tc##1{\textcolor[rgb]{0.73,0.40,0.13}{##1}}}
\expandafter\def\csname PY@tok@sr\endcsname{\def\PY@tc##1{\textcolor[rgb]{0.73,0.40,0.53}{##1}}}
\expandafter\def\csname PY@tok@ss\endcsname{\def\PY@tc##1{\textcolor[rgb]{0.10,0.09,0.49}{##1}}}
\expandafter\def\csname PY@tok@sx\endcsname{\def\PY@tc##1{\textcolor[rgb]{0.00,0.50,0.00}{##1}}}
\expandafter\def\csname PY@tok@m\endcsname{\def\PY@tc##1{\textcolor[rgb]{0.40,0.40,0.40}{##1}}}
\expandafter\def\csname PY@tok@gh\endcsname{\let\PY@bf=\textbf\def\PY@tc##1{\textcolor[rgb]{0.00,0.00,0.50}{##1}}}
\expandafter\def\csname PY@tok@gu\endcsname{\let\PY@bf=\textbf\def\PY@tc##1{\textcolor[rgb]{0.50,0.00,0.50}{##1}}}
\expandafter\def\csname PY@tok@gd\endcsname{\def\PY@tc##1{\textcolor[rgb]{0.63,0.00,0.00}{##1}}}
\expandafter\def\csname PY@tok@gi\endcsname{\def\PY@tc##1{\textcolor[rgb]{0.00,0.63,0.00}{##1}}}
\expandafter\def\csname PY@tok@gr\endcsname{\def\PY@tc##1{\textcolor[rgb]{1.00,0.00,0.00}{##1}}}
\expandafter\def\csname PY@tok@ge\endcsname{\let\PY@it=\textit}
\expandafter\def\csname PY@tok@gs\endcsname{\let\PY@bf=\textbf}
\expandafter\def\csname PY@tok@gp\endcsname{\let\PY@bf=\textbf\def\PY@tc##1{\textcolor[rgb]{0.00,0.00,0.50}{##1}}}
\expandafter\def\csname PY@tok@go\endcsname{\def\PY@tc##1{\textcolor[rgb]{0.53,0.53,0.53}{##1}}}
\expandafter\def\csname PY@tok@gt\endcsname{\def\PY@tc##1{\textcolor[rgb]{0.00,0.27,0.87}{##1}}}
\expandafter\def\csname PY@tok@err\endcsname{\def\PY@bc##1{\setlength{\fboxsep}{0pt}\fcolorbox[rgb]{1.00,0.00,0.00}{1,1,1}{\strut ##1}}}
\expandafter\def\csname PY@tok@kc\endcsname{\let\PY@bf=\textbf\def\PY@tc##1{\textcolor[rgb]{0.00,0.50,0.00}{##1}}}
\expandafter\def\csname PY@tok@kd\endcsname{\let\PY@bf=\textbf\def\PY@tc##1{\textcolor[rgb]{0.00,0.50,0.00}{##1}}}
\expandafter\def\csname PY@tok@kn\endcsname{\let\PY@bf=\textbf\def\PY@tc##1{\textcolor[rgb]{0.00,0.50,0.00}{##1}}}
\expandafter\def\csname PY@tok@kr\endcsname{\let\PY@bf=\textbf\def\PY@tc##1{\textcolor[rgb]{0.00,0.50,0.00}{##1}}}
\expandafter\def\csname PY@tok@bp\endcsname{\def\PY@tc##1{\textcolor[rgb]{0.00,0.50,0.00}{##1}}}
\expandafter\def\csname PY@tok@fm\endcsname{\def\PY@tc##1{\textcolor[rgb]{0.00,0.00,1.00}{##1}}}
\expandafter\def\csname PY@tok@vc\endcsname{\def\PY@tc##1{\textcolor[rgb]{0.10,0.09,0.49}{##1}}}
\expandafter\def\csname PY@tok@vg\endcsname{\def\PY@tc##1{\textcolor[rgb]{0.10,0.09,0.49}{##1}}}
\expandafter\def\csname PY@tok@vi\endcsname{\def\PY@tc##1{\textcolor[rgb]{0.10,0.09,0.49}{##1}}}
\expandafter\def\csname PY@tok@vm\endcsname{\def\PY@tc##1{\textcolor[rgb]{0.10,0.09,0.49}{##1}}}
\expandafter\def\csname PY@tok@sa\endcsname{\def\PY@tc##1{\textcolor[rgb]{0.73,0.13,0.13}{##1}}}
\expandafter\def\csname PY@tok@sb\endcsname{\def\PY@tc##1{\textcolor[rgb]{0.73,0.13,0.13}{##1}}}
\expandafter\def\csname PY@tok@sc\endcsname{\def\PY@tc##1{\textcolor[rgb]{0.73,0.13,0.13}{##1}}}
\expandafter\def\csname PY@tok@dl\endcsname{\def\PY@tc##1{\textcolor[rgb]{0.73,0.13,0.13}{##1}}}
\expandafter\def\csname PY@tok@s2\endcsname{\def\PY@tc##1{\textcolor[rgb]{0.73,0.13,0.13}{##1}}}
\expandafter\def\csname PY@tok@sh\endcsname{\def\PY@tc##1{\textcolor[rgb]{0.73,0.13,0.13}{##1}}}
\expandafter\def\csname PY@tok@s1\endcsname{\def\PY@tc##1{\textcolor[rgb]{0.73,0.13,0.13}{##1}}}
\expandafter\def\csname PY@tok@mb\endcsname{\def\PY@tc##1{\textcolor[rgb]{0.40,0.40,0.40}{##1}}}
\expandafter\def\csname PY@tok@mf\endcsname{\def\PY@tc##1{\textcolor[rgb]{0.40,0.40,0.40}{##1}}}
\expandafter\def\csname PY@tok@mh\endcsname{\def\PY@tc##1{\textcolor[rgb]{0.40,0.40,0.40}{##1}}}
\expandafter\def\csname PY@tok@mi\endcsname{\def\PY@tc##1{\textcolor[rgb]{0.40,0.40,0.40}{##1}}}
\expandafter\def\csname PY@tok@il\endcsname{\def\PY@tc##1{\textcolor[rgb]{0.40,0.40,0.40}{##1}}}
\expandafter\def\csname PY@tok@mo\endcsname{\def\PY@tc##1{\textcolor[rgb]{0.40,0.40,0.40}{##1}}}
\expandafter\def\csname PY@tok@ch\endcsname{\let\PY@it=\textit\def\PY@tc##1{\textcolor[rgb]{0.25,0.50,0.50}{##1}}}
\expandafter\def\csname PY@tok@cm\endcsname{\let\PY@it=\textit\def\PY@tc##1{\textcolor[rgb]{0.25,0.50,0.50}{##1}}}
\expandafter\def\csname PY@tok@cpf\endcsname{\let\PY@it=\textit\def\PY@tc##1{\textcolor[rgb]{0.25,0.50,0.50}{##1}}}
\expandafter\def\csname PY@tok@c1\endcsname{\let\PY@it=\textit\def\PY@tc##1{\textcolor[rgb]{0.25,0.50,0.50}{##1}}}
\expandafter\def\csname PY@tok@cs\endcsname{\let\PY@it=\textit\def\PY@tc##1{\textcolor[rgb]{0.25,0.50,0.50}{##1}}}

\def\PYZbs{\char`\\}
\def\PYZus{\char`\_}
\def\PYZob{\char`\{}
\def\PYZcb{\char`\}}
\def\PYZca{\char`\^}
\def\PYZam{\char`\&}
\def\PYZlt{\char`\<}
\def\PYZgt{\char`\>}
\def\PYZsh{\char`\#}
\def\PYZpc{\char`\%}
\def\PYZdl{\char`\$}
\def\PYZhy{\char`\-}
\def\PYZsq{\char`\'}
\def\PYZdq{\char`\"}
\def\PYZti{\char`\~}
% for compatibility with earlier versions
\def\PYZat{@}
\def\PYZlb{[}
\def\PYZrb{]}
\makeatother

% ANSI colors
\definecolor{ansi-black}{HTML}{3E424D}
\definecolor{ansi-black-intense}{HTML}{282C36}
\definecolor{ansi-red}{HTML}{E75C58}
\definecolor{ansi-red-intense}{HTML}{B22B31}
\definecolor{ansi-green}{HTML}{00A250}
\definecolor{ansi-green-intense}{HTML}{007427}
\definecolor{ansi-yellow}{HTML}{DDB62B}
\definecolor{ansi-yellow-intense}{HTML}{B27D12}
\definecolor{ansi-blue}{HTML}{208FFB}
\definecolor{ansi-blue-intense}{HTML}{0065CA}
\definecolor{ansi-magenta}{HTML}{D160C4}
\definecolor{ansi-magenta-intense}{HTML}{A03196}
\definecolor{ansi-cyan}{HTML}{60C6C8}
\definecolor{ansi-cyan-intense}{HTML}{258F8F}
\definecolor{ansi-white}{HTML}{C5C1B4}
\definecolor{ansi-white-intense}{HTML}{A1A6B2}

% commands and environments needed by pandoc snippets
% extracted from the output of `pandoc -s`
\providecommand{\tightlist}{%
  \setlength{\itemsep}{0pt}\setlength{\parskip}{0pt}}
\DefineVerbatimEnvironment{Highlighting}{Verbatim}{commandchars=\\\{\}}
% Add ',fontsize=\small' for more characters per line
\newenvironment{Shaded}{}{}
\newcommand{\KeywordTok}[1]{\textcolor[rgb]{0.00,0.44,0.13}{\textbf{{#1}}}}
\newcommand{\DataTypeTok}[1]{\textcolor[rgb]{0.56,0.13,0.00}{{#1}}}
\newcommand{\DecValTok}[1]{\textcolor[rgb]{0.25,0.63,0.44}{{#1}}}
\newcommand{\BaseNTok}[1]{\textcolor[rgb]{0.25,0.63,0.44}{{#1}}}
\newcommand{\FloatTok}[1]{\textcolor[rgb]{0.25,0.63,0.44}{{#1}}}
\newcommand{\CharTok}[1]{\textcolor[rgb]{0.25,0.44,0.63}{{#1}}}
\newcommand{\StringTok}[1]{\textcolor[rgb]{0.25,0.44,0.63}{{#1}}}
\newcommand{\CommentTok}[1]{\textcolor[rgb]{0.38,0.63,0.69}{\textit{{#1}}}}
\newcommand{\OtherTok}[1]{\textcolor[rgb]{0.00,0.44,0.13}{{#1}}}
\newcommand{\AlertTok}[1]{\textcolor[rgb]{1.00,0.00,0.00}{\textbf{{#1}}}}
\newcommand{\FunctionTok}[1]{\textcolor[rgb]{0.02,0.16,0.49}{{#1}}}
\newcommand{\RegionMarkerTok}[1]{{#1}}
\newcommand{\ErrorTok}[1]{\textcolor[rgb]{1.00,0.00,0.00}{\textbf{{#1}}}}
\newcommand{\NormalTok}[1]{{#1}}

% Additional commands for more recent versions of Pandoc
\newcommand{\ConstantTok}[1]{\textcolor[rgb]{0.53,0.00,0.00}{{#1}}}
\newcommand{\SpecialCharTok}[1]{\textcolor[rgb]{0.25,0.44,0.63}{{#1}}}
\newcommand{\VerbatimStringTok}[1]{\textcolor[rgb]{0.25,0.44,0.63}{{#1}}}
\newcommand{\SpecialStringTok}[1]{\textcolor[rgb]{0.73,0.40,0.53}{{#1}}}
\newcommand{\ImportTok}[1]{{#1}}
\newcommand{\DocumentationTok}[1]{\textcolor[rgb]{0.73,0.13,0.13}{\textit{{#1}}}}
\newcommand{\AnnotationTok}[1]{\textcolor[rgb]{0.38,0.63,0.69}{\textbf{\textit{{#1}}}}}
\newcommand{\CommentVarTok}[1]{\textcolor[rgb]{0.38,0.63,0.69}{\textbf{\textit{{#1}}}}}
\newcommand{\VariableTok}[1]{\textcolor[rgb]{0.10,0.09,0.49}{{#1}}}
\newcommand{\ControlFlowTok}[1]{\textcolor[rgb]{0.00,0.44,0.13}{\textbf{{#1}}}}
\newcommand{\OperatorTok}[1]{\textcolor[rgb]{0.40,0.40,0.40}{{#1}}}
\newcommand{\BuiltInTok}[1]{{#1}}
\newcommand{\ExtensionTok}[1]{{#1}}
\newcommand{\PreprocessorTok}[1]{\textcolor[rgb]{0.74,0.48,0.00}{{#1}}}
\newcommand{\AttributeTok}[1]{\textcolor[rgb]{0.49,0.56,0.16}{{#1}}}
\newcommand{\InformationTok}[1]{\textcolor[rgb]{0.38,0.63,0.69}{\textbf{\textit{{#1}}}}}
\newcommand{\WarningTok}[1]{\textcolor[rgb]{0.38,0.63,0.69}{\textbf{\textit{{#1}}}}}

% Define a nice break command that doesn't care if a line doesn't already
% exist.
\def\br{\hspace*{\fill} \\* }

% Math Jax compatability definitions
\def\gt{>}
\def\lt{<}

\setcounter{secnumdepth}{5}

% Colors for the hyperref package
\definecolor{urlcolor}{rgb}{0,.145,.698}
\definecolor{linkcolor}{rgb}{.71,0.21,0.01}
\definecolor{citecolor}{rgb}{.12,.54,.11}

\DeclareTranslationFallback{Author}{Author}
\DeclareTranslation{Portuges}{Author}{Autor}

\DeclareTranslationFallback{List of Codes}{List of Codes}
\DeclareTranslation{Catalan}{List of Codes}{Llista de Codis}
\DeclareTranslation{Danish}{List of Codes}{Liste over Koder}
\DeclareTranslation{German}{List of Codes}{Liste der Codes}
\DeclareTranslation{Spanish}{List of Codes}{Lista de C\'{o}digos}
\DeclareTranslation{French}{List of Codes}{Liste des Codes}
\DeclareTranslation{Italian}{List of Codes}{Elenco dei Codici}
\DeclareTranslation{Dutch}{List of Codes}{Lijst van Codes}
\DeclareTranslation{Portuges}{List of Codes}{Lista de C\'{o}digos}

\DeclareTranslationFallback{Supervisors}{Supervisors}
\DeclareTranslation{Catalan}{Supervisors}{Supervisors}
\DeclareTranslation{Danish}{Supervisors}{Vejledere}
\DeclareTranslation{German}{Supervisors}{Vorgesetzten}
\DeclareTranslation{Spanish}{Supervisors}{Supervisores}
\DeclareTranslation{French}{Supervisors}{Superviseurs}
\DeclareTranslation{Italian}{Supervisors}{Le autorit\`{a} di vigilanza}
\DeclareTranslation{Dutch}{Supervisors}{supervisors}
\DeclareTranslation{Portuguese}{Supervisors}{Supervisores}

\definecolor{codegreen}{rgb}{0,0.6,0}
\definecolor{codegray}{rgb}{0.5,0.5,0.5}
\definecolor{codepurple}{rgb}{0.58,0,0.82}
\definecolor{backcolour}{rgb}{0.95,0.95,0.95}

\lstdefinestyle{mystyle}{
    commentstyle=\color{codegreen},
    keywordstyle=\color{magenta},
    numberstyle=\tiny\color{codegray},
    stringstyle=\color{codepurple},
    basicstyle=\ttfamily,
    breakatwhitespace=false,
    keepspaces=true,
    numbers=left,
    numbersep=10pt,
    showspaces=false,
    showstringspaces=false,
    showtabs=false,
    tabsize=2,
    breaklines=true,
    literate={\-}{}{0\discretionary{-}{}{-}},
  postbreak=\mbox{\textcolor{red}{$\hookrightarrow$}\space},
}

\lstset{style=mystyle}

\surroundwithmdframed[
  hidealllines=true,
  backgroundcolor=backcolour,
  innerleftmargin=0pt,
  innerrightmargin=0pt,
  innertopmargin=0pt,
  innerbottommargin=0pt]{lstlisting}

%%%%%%%%%%%%

%%%%%%%%%%%% MARGINS

 % Used to adjust the document margins
\usepackage{geometry}
\geometry{tmargin=1in,bmargin=1in,lmargin=1in,rmargin=1in,
nohead,includefoot,footskip=25pt}
% you can use showframe option to check the margins visually
%%%%%%%%%%%%

%%%%%%%%%%%% COMMANDS

% ensure new section starts on new page
\addtokomafont{section}{\clearpage}

% Prevent overflowing lines due to hard-to-break entities
\sloppy

% Setup hyperref package
\hypersetup{
    breaklinks=true,  % so long urls are correctly broken across lines
    colorlinks=true,
    urlcolor=urlcolor,
    linkcolor=linkcolor,
    citecolor=citecolor,
    }

% ensure figures are placed within subsections
\makeatletter
\AtBeginDocument{%
    \expandafter\renewcommand\expandafter\subsection\expandafter
    {\expandafter\@fb@secFB\subsection}%
    \newcommand\@fb@secFB{\FloatBarrier
    \gdef\@fb@afterHHook{\@fb@topbarrier \gdef\@fb@afterHHook{}}}%
    \g@addto@macro\@afterheading{\@fb@afterHHook}%
    \gdef\@fb@afterHHook{}%
}
\makeatother

% number figures, tables and equations by section
\counterwithout{figure}{section}
\counterwithout{table}{section}
\counterwithout{equation}{section}
\makeatletter
\@addtoreset{table}{section}
\@addtoreset{figure}{section}
\@addtoreset{equation}{section}
\makeatother
\renewcommand\thetable{\thesection.\arabic{table}}
\renewcommand\thefigure{\thesection.\arabic{figure}}
\renewcommand\theequation{\thesection.\arabic{equation}}

    % set global options for float placement
    \makeatletter
        \providecommand*\setfloatlocations[2]{\@namedef{fps@#1}{#2}}
    \makeatother

% align captions to left (indented)
\captionsetup{justification=raggedright,
singlelinecheck=false,format=hang,labelfont={it,bf}}

% shift footer down so space between separation line
\ModifyLayer[addvoffset=.6ex]{scrheadings.foot.odd}
\ModifyLayer[addvoffset=.6ex]{scrheadings.foot.even}
\ModifyLayer[addvoffset=.6ex]{scrheadings.foot.oneside}
\ModifyLayer[addvoffset=.6ex]{plain.scrheadings.foot.odd}
\ModifyLayer[addvoffset=.6ex]{plain.scrheadings.foot.even}
\ModifyLayer[addvoffset=.6ex]{plain.scrheadings.foot.oneside}
\pagestyle{scrheadings}
\clearscrheadfoot{}
\ifoot{\leftmark}
\renewcommand{\sectionmark}[1]{\markleft{\thesection\ #1}}
\ofoot{\pagemark}
\cfoot{}

%%%%%%%%%%%%

%%%%%%%%%%%% FINAL HEADER MATERIAL

% clereref must be loaded after anything that changes the referencing system
\usepackage{cleveref}
\creflabelformat{equation}{#2#1#3}

% make the code float work with cleverref
\crefname{codecell}{code}{codes}
\Crefname{codecell}{code}{codes}
% make the text float work with cleverref
\crefname{textcell}{text}{texts}
\Crefname{textcell}{text}{texts}
% make the text float work with cleverref
\crefname{errorcell}{error}{errors}
\Crefname{errorcell}{error}{errors}

%%%%%%%%%%%%

\begin{document}

    \begin{titlepage}

  \begin{center}

  \vspace*{1cm}

  \Huge\textbf{Overpressure Analysis of Girard Point Refinery Accident}

  \vspace{0.5cm}\LARGE{Boiling-Liquid Expanding-Vapor Explosion}

  \vspace{1.5cm}

  \begin{minipage}{0.8\textwidth}
    \begin{center}
    \begin{minipage}{0.39\textwidth}
    \begin{flushleft} \Large
    \emph{\GetTranslation{Author}:}\\S. Kevin McNeill\\\href{mailto:shonn.mcneill@atf.gov}{shonn.mcneill@atf.gov}
    \end{flushleft}
    \end{minipage}
    \hspace{\fill}
    \begin{minipage}{0.39\textwidth}
    \begin{flushright} \Large
    \end{flushright}
    \end{minipage}
    \end{center}
  \end{minipage}

  \vfill

  \begin{minipage}{0.8\textwidth}
  \begin{center}
  \end{center}
  \end{minipage}

  \vspace{0.8cm}
      \LARGE{National Center for Explosives Training and Research}\\
      \LARGE{Explosives Research and Development Division}\\

  \vspace{0.4cm}

  \today

  \end{center}
  \end{titlepage}

    \begingroup
    \let\cleardoublepage\relax
    \let\clearpage\relax\tableofcontents
    \endgroup

\hypertarget{introduction}{%
\section{Introduction}\label{introduction}}

On June 21, 2019, at approximately 4:22 AM, a fire and explosion, see
\cref{fig:fire}, occurred at the Girard Point Refinery owned by
Philadelphia Energy Solutions \cite{Renshaw2019}. One of the three
explosions observed that day was from the V1 treater-feed-surge-drum in
the refineries pretreatment unit. The explosion propelled the largest
piece (estimated: \(41809.6lb\:(18964.5kg)\)) of the drum approximately
\(2100ft\:(640m)\) from the blast seat, see \cref{fig:tank}. It is
hypothesized that a boiling-liquid expanding-vapor explosion (BLEVE)
event provided the energy to generate the blast wave. The Philadelphia
Fire Department requested ATF estimate the blast overpressure generated
when the tank exploded. This paper is an engineering analysis to
estimate the blast overpressure assuming a BLEVE occurred. The analysis
is based upon an adiabatic and isentropic energy analysis developed by
the Center for Chemical Process Safety \cite{Safety2010}.

\begin{figure}[H]
\hypertarget{fig:fire}{%
\begin{center}
\adjustimage{max size={0.9\linewidth}{0.9\paperheight},width=0.75\linewidth}{refinery_bleve_files/fig_refinery_fire.png}
\end{center}
\caption{A view of the Philadelphia Energy Solutions Inc's oil refinery while on
fire \cite{Maykuth2019}.}\label{fig:fire}
}
\end{figure}

    \begin{figure}[H]\begin{center}\adjustimage{max size={0.9\linewidth}{0.9\paperheight}}{refinery_bleve_files/output_2_0.png}\end{center}\caption{Largest piece of the treater-feed-surge-drum was propelled 2100ft from
the blast seat. The upper image shows the drum piece in flight moving
left to right. The second image shows the drum piece after striking the
edge of the Schuylkill River shoreline.}\label{fig:tank}\end{figure}

\hypertarget{background}{%
\section{Background}\label{background}}

\hypertarget{refinery}{%
\subsection{Refinery}\label{refinery}}

The Girard Point Refinery is located in southwest Philadelphia, PA, on
the Schuylkill River, see \cref{fig:map}. The refinery produced
approximately 335,000 bpd of gasoline, and was the largest on the East
Coast \cite{AssociatedPress2019}. The treater-feed-surge-drum, involved
in the explosion, is part of the pretreatment process of alkylation used
in the production of gasoline.

\begin{figure}[H]
\hypertarget{fig:map}{%
\begin{center}
\adjustimage{max size={0.9\linewidth}{0.9\paperheight},width=0.75\linewidth}{refinery_bleve_files/fig_map_girard.jpeg}
\end{center}
\caption{Map of Girard Point Refinery showing the blast seat location
\cite{Duchneskie2019}.}\label{fig:map}
}
\end{figure}

\hypertarget{treater-feed-surge-tank}{%
\subsection{Treater Feed Surge Tank}\label{treater-feed-surge-tank}}

Alkylaition generally converts propylene \((C_3H_6)\), butylene
\((C_4H_8)\), pentene \((C_5H_{10})\), and isobutane \((C_3H_{10})\) to
alkane liquids such as isoheptane \((C_7H_{16})\) and isooctane
\((C_8H_{16})\). These alkylates are a highly valued component in the
production of gasoline because of there high octane and low vapor
pressure \cite{flowserve2000}. The treater-feed-surge-drum (TFSD) was
located between the fluid catalytic cracker and the alkylaition unit.
The purpose of a surge drum is to stabilize fluctuations in the overall
system flow rate. The TFSD was part of the pretreatment process for
alkylation, see \cref{fig:fig_process_flow}. During pretreatment, also
referred to as sweetening, sulfur compounds (hydrogen sulfide, thiophene
and mercaptan) are removed to improve color, odor, and oxidation
stability.

\begin{figure}[H]
\hypertarget{fig:fig_process_flow}{%
\begin{center}
\adjustimage{max size={0.9\linewidth}{0.9\paperheight},width=0.75\linewidth}{refinery_bleve_files/fig_process_flow.jpeg}
\end{center}
\caption{Simple flow diagram of the sweetening and treating process where the
TFSD was located at the time of the explosion.
\cite{Temur2014,Malone2019}}\label{fig:fig_process_flow}
}
\end{figure}

The TFSD tank measured 39'-8" in length, not including the heads, see
\cref{tbl:tbl_tank} and \cref{fig:fig_tank} for construction details
\cite{PES2019}. At the time of the explosion, the TFSD contained
\(20160gal\:(76.3m^3)\) of butane (50\% by volume) and butene (40\% by
volume) and other lesser constituients, see
\cref{tbl:tbl_chemicals_in_tank}.

\begin{longtable}[]{@{}rll@{}}
\caption{Treater-Feed-Surge-Drum Construction Parameters \cite{PES2019}
\label{tbl:tbl_tank}}\tabularnewline
\toprule
\begin{minipage}[b]{0.23\columnwidth}\raggedleft
Parameter\strut
\end{minipage} & \begin{minipage}[b]{0.23\columnwidth}\raggedright
Value\strut
\end{minipage} & \begin{minipage}[b]{0.23\columnwidth}\raggedright
Units\strut
\end{minipage}\tabularnewline
\midrule
\endfirsthead
\toprule
\begin{minipage}[b]{0.23\columnwidth}\raggedleft
Parameter\strut
\end{minipage} & \begin{minipage}[b]{0.23\columnwidth}\raggedright
Value\strut
\end{minipage} & \begin{minipage}[b]{0.23\columnwidth}\raggedright
Units\strut
\end{minipage}\tabularnewline
\midrule
\endhead
\begin{minipage}[t]{0.23\columnwidth}\raggedleft
Type\strut
\end{minipage} & \begin{minipage}[t]{0.23\columnwidth}\raggedright
horizontal\strut
\end{minipage} & \begin{minipage}[t]{0.23\columnwidth}\raggedright
NA\strut
\end{minipage}\tabularnewline
\begin{minipage}[t]{0.23\columnwidth}\raggedleft
Year Built\strut
\end{minipage} & \begin{minipage}[t]{0.23\columnwidth}\raggedright
1972\strut
\end{minipage} & \begin{minipage}[t]{0.23\columnwidth}\raggedright
NA\strut
\end{minipage}\tabularnewline
\begin{minipage}[t]{0.23\columnwidth}\raggedleft
Construction Material\strut
\end{minipage} & \begin{minipage}[t]{0.23\columnwidth}\raggedright
A516 Type 70 Steel\strut
\end{minipage} & \begin{minipage}[t]{0.23\columnwidth}\raggedright
NA\strut
\end{minipage}\tabularnewline
\begin{minipage}[t]{0.23\columnwidth}\raggedleft
Tank Wall Thickness\strut
\end{minipage} & \begin{minipage}[t]{0.23\columnwidth}\raggedright
0.8125\strut
\end{minipage} & \begin{minipage}[t]{0.23\columnwidth}\raggedright
in\strut
\end{minipage}\tabularnewline
\begin{minipage}[t]{0.23\columnwidth}\raggedleft
Volume\strut
\end{minipage} & \begin{minipage}[t]{0.23\columnwidth}\raggedright
372228\strut
\end{minipage} & \begin{minipage}[t]{0.23\columnwidth}\raggedright
gal\strut
\end{minipage}\tabularnewline
\begin{minipage}[t]{0.23\columnwidth}\raggedleft
Percent Filled\strut
\end{minipage} & \begin{minipage}[t]{0.23\columnwidth}\raggedright
53.3\strut
\end{minipage} & \begin{minipage}[t]{0.23\columnwidth}\raggedright
\%\strut
\end{minipage}\tabularnewline
\begin{minipage}[t]{0.23\columnwidth}\raggedleft
Mass\strut
\end{minipage} & \begin{minipage}[t]{0.23\columnwidth}\raggedright
74660\strut
\end{minipage} & \begin{minipage}[t]{0.23\columnwidth}\raggedright
lb\strut
\end{minipage}\tabularnewline
\begin{minipage}[t]{0.23\columnwidth}\raggedleft
Safety Valve Set\strut
\end{minipage} & \begin{minipage}[t]{0.23\columnwidth}\raggedright
155\strut
\end{minipage} & \begin{minipage}[t]{0.23\columnwidth}\raggedright
psig\strut
\end{minipage}\tabularnewline
\bottomrule
\end{longtable}

\begin{figure}[H]
\hypertarget{fig:fig_tank}{%
\begin{center}
\adjustimage{max size={0.9\linewidth}{0.9\paperheight},width=0.8\linewidth}{refinery_bleve_files/fig_tank.jpeg}
\end{center}
\caption{Diagram depicting the dimensions of the TFSD tank. The tank was
positioned 25 ft above ground level and estimated to contain 20160 gal
(53.3\% filled) of butane (50\% by volume) and butene (40\% by volume)
at the time of the explosion.\cite{PES2019}}\label{fig:fig_tank}
}
\end{figure}

\begin{longtable}[]{@{}rr@{}}
\caption{Chemical Contents in the Treater-Feed-Surge-Drum Nearest to the
Time of the Explosion \cite{PES2019}
\label{tbl:tbl_chemicals_in_tank}}\tabularnewline
\toprule
\begin{minipage}[b]{0.19\columnwidth}\raggedleft
Chemical\strut
\end{minipage} & \begin{minipage}[b]{0.19\columnwidth}\raggedleft
Percent by Volume\strut
\end{minipage}\tabularnewline
\midrule
\endfirsthead
\toprule
\begin{minipage}[b]{0.19\columnwidth}\raggedleft
Chemical\strut
\end{minipage} & \begin{minipage}[b]{0.19\columnwidth}\raggedleft
Percent by Volume\strut
\end{minipage}\tabularnewline
\midrule
\endhead
\begin{minipage}[t]{0.19\columnwidth}\raggedleft
methane\strut
\end{minipage} & \begin{minipage}[t]{0.19\columnwidth}\raggedleft
0.01\strut
\end{minipage}\tabularnewline
\begin{minipage}[t]{0.19\columnwidth}\raggedleft
ethylene\strut
\end{minipage} & \begin{minipage}[t]{0.19\columnwidth}\raggedleft
0.00\strut
\end{minipage}\tabularnewline
\begin{minipage}[t]{0.19\columnwidth}\raggedleft
ethane\strut
\end{minipage} & \begin{minipage}[t]{0.19\columnwidth}\raggedleft
0.01\strut
\end{minipage}\tabularnewline
\begin{minipage}[t]{0.19\columnwidth}\raggedleft
propane\strut
\end{minipage} & \begin{minipage}[t]{0.19\columnwidth}\raggedleft
0.90\strut
\end{minipage}\tabularnewline
\begin{minipage}[t]{0.19\columnwidth}\raggedleft
propylene\strut
\end{minipage} & \begin{minipage}[t]{0.19\columnwidth}\raggedleft
0.10\strut
\end{minipage}\tabularnewline
\begin{minipage}[t]{0.19\columnwidth}\raggedleft
isobutane\strut
\end{minipage} & \begin{minipage}[t]{0.19\columnwidth}\raggedleft
37.28\strut
\end{minipage}\tabularnewline
\begin{minipage}[t]{0.19\columnwidth}\raggedleft
nbutane\strut
\end{minipage} & \begin{minipage}[t]{0.19\columnwidth}\raggedleft
12.81\strut
\end{minipage}\tabularnewline
\begin{minipage}[t]{0.19\columnwidth}\raggedleft
butens\strut
\end{minipage} & \begin{minipage}[t]{0.19\columnwidth}\raggedleft
40.41\strut
\end{minipage}\tabularnewline
\begin{minipage}[t]{0.19\columnwidth}\raggedleft
neopentane\strut
\end{minipage} & \begin{minipage}[t]{0.19\columnwidth}\raggedleft
0.00\strut
\end{minipage}\tabularnewline
\begin{minipage}[t]{0.19\columnwidth}\raggedleft
isopentane\strut
\end{minipage} & \begin{minipage}[t]{0.19\columnwidth}\raggedleft
3.94\strut
\end{minipage}\tabularnewline
\begin{minipage}[t]{0.19\columnwidth}\raggedleft
npentane\strut
\end{minipage} & \begin{minipage}[t]{0.19\columnwidth}\raggedleft
0.25\strut
\end{minipage}\tabularnewline
\begin{minipage}[t]{0.19\columnwidth}\raggedleft
butadiene\strut
\end{minipage} & \begin{minipage}[t]{0.19\columnwidth}\raggedleft
0.33\strut
\end{minipage}\tabularnewline
\begin{minipage}[t]{0.19\columnwidth}\raggedleft
benzene\strut
\end{minipage} & \begin{minipage}[t]{0.19\columnwidth}\raggedleft
0.00\strut
\end{minipage}\tabularnewline
\begin{minipage}[t]{0.19\columnwidth}\raggedleft
C5 olefins\strut
\end{minipage} & \begin{minipage}[t]{0.19\columnwidth}\raggedleft
3.54\strut
\end{minipage}\tabularnewline
\begin{minipage}[t]{0.19\columnwidth}\raggedleft
C6 sats\strut
\end{minipage} & \begin{minipage}[t]{0.19\columnwidth}\raggedleft
0.33\strut
\end{minipage}\tabularnewline
\begin{minipage}[t]{0.19\columnwidth}\raggedleft
C7+\strut
\end{minipage} & \begin{minipage}[t]{0.19\columnwidth}\raggedleft
0.04\strut
\end{minipage}\tabularnewline
\bottomrule
\end{longtable}

For this analysis only the butenes and butanes will be considered as
they make up more than 90\% of the fluid volume. The boiling
temperatures for butene and butane are
\(20.66\: ^{\circ} F\:(-6.3\: ^{\circ} C)\) and
\(30.2\: ^{\circ} F\:(-1.0\: ^{\circ} C)\) respectively; both are well
above the tank temperature when exposed to fire so a BLEVE assumptions
is reasonable.

\hypertarget{pressure-relief-valve}{%
\subsection{Pressure Relief Valve}\label{pressure-relief-valve}}

The TFSD was fitted with a Consolidated (1906-30LC-1-CC-MS-31-RF-1) 3" x
4" pressure relief valve (PRV). The relief pressure was set to 155 psig
and the relief temperature was set to 183.5 \(^{\circ} F\). The PRV was
positioned on the top of the TRFD, see \cref{fig:fig_tank}.

\hypertarget{recovered-drum-debris}{%
\subsection{Recovered Drum Debris}\label{recovered-drum-debris}}

Three major pieces of the TFSD were identified after the explosion. They
were thrown a maximum of \(2100ft\:(640m)\), see
\cref{fig:blast_debris}. The largest piece (Large End Cap) was
approximately \(22ft\:(6.7m)\) in length, see \cref{fig:large_end_cap},
or a little more than half the original tank volume. The other end of
the TFSD (Small End Cap) was recovered \(1761ft\:(536m)\) from the blast
seat. It was approximately \(5ft\:(1.5m)\) in length, see
\cref{fig:small_end_cap}. The piece thrown the shortest distance
(fillet) was \(819ft\:(249m)\) from the blast seat. The fillet piece was
heavily damaged and a photographic analysis of the length was not
possible. However, based on the original length of the tank and removing
the large and small end cap lengths the fillet length is approximately
\(12ft\:(3.6m)\), see \cref{fig:fillet}.

\begin{figure}[H]
\hypertarget{fig:blast_debris}{%
\begin{center}
\adjustimage{max size={0.9\linewidth}{0.9\paperheight},width=0.75\linewidth}{refinery_bleve_files/drum_debris_and_blast_seat_distances_small.png}
\end{center}
\caption{Map of Girard Point showing the blast seat and locations of the three
main pieces of the TSFD (V-1).\cite{Malone2019a}}\label{fig:blast_debris}
}
\end{figure}

\begin{figure}[H]
\hypertarget{fig:large_end_cap}{%
\begin{center}
\adjustimage{max size={0.9\linewidth}{0.9\paperheight},width=0.75\linewidth}{refinery_bleve_files/tank_length_estimate.png}
\end{center}
\caption{Photograph of the large end cap, estimated to be \(22ft\:(6.7m)\) in
length based on photographic analysis. In this photograph the tank has
been moved from it's original landing location.\cite{Malone2019a}}\label{fig:large_end_cap}
}
\end{figure}

\begin{figure}[H]
\hypertarget{fig:small_end_cap}{%
\begin{center}
\adjustimage{max size={0.9\linewidth}{0.9\paperheight},width=0.75\linewidth}{refinery_bleve_files/end_cap_03.jpg}
\end{center}
\caption{Photograph of the small end cap, estimated to be \(5ft\:(1.5m)\) in
length based on photographic analysis.\cite{Malone2019a}}\label{fig:small_end_cap}
}
\end{figure}

\begin{figure}[H]
\hypertarget{fig:fillet}{%
\begin{center}
\adjustimage{max size={0.9\linewidth}{0.9\paperheight},width=0.75\linewidth}{refinery_bleve_files/1_fish_fillet2.jpg}
\end{center}
\caption{Photograph of the fish fillet piece, calculated to be \(12ft\:(3.6m)\)
in length based on the total length of the drum less the lengths of the
large and small end caps.\cite{Malone2019a}}\label{fig:fillet}
}
\end{figure}

The calculated masses of each piece of the TFSD without appurtanaces are
summarized in \cref{tbl:tbl_tank}. The mass calculations are based on
the following,

Ellipsoidal Head:
\begin{equation}V_{eh}=\frac{\pi D^2 (D/4)}{6}\end{equation} Cylinder:
\begin{equation}V_c = \frac{\pi D^2 l}{4}\end{equation}

with a drum wall thickness of \(0.83in\:(2.1cm)\), a tank diameter of
\(12ft\:(3.7m)\), and a material density for A516 steel of \(7.8g/cc\).
Refinery records estimate the weight of the vessel empty as
\(74660lb\:(33865kg)\) \cite{Malone2019}.

\begin{longtable}[]{@{}lrrrr@{}}
\caption{Estimated Treater-Feed-Surge-Drum Debris Mass \cite{PES2019}
\label{tbl:tbl_tank}}\tabularnewline
\toprule
\begin{minipage}[b]{0.17\columnwidth}\raggedright
TFSD ID\strut
\end{minipage} & \begin{minipage}[b]{0.09\columnwidth}\raggedleft
Cylinder (ft)\strut
\end{minipage} & \begin{minipage}[b]{0.09\columnwidth}\raggedleft
Calculated Debris Mass (lb)\strut
\end{minipage} & \begin{minipage}[b]{0.13\columnwidth}\raggedleft
Ratio Calculated (Debris/Total)\strut
\end{minipage} & \begin{minipage}[b]{0.13\columnwidth}\raggedleft
Refinery Recorded Mass (lb)\strut
\end{minipage}\tabularnewline
\midrule
\endfirsthead
\toprule
\begin{minipage}[b]{0.17\columnwidth}\raggedright
TFSD ID\strut
\end{minipage} & \begin{minipage}[b]{0.09\columnwidth}\raggedleft
Cylinder (ft)\strut
\end{minipage} & \begin{minipage}[b]{0.09\columnwidth}\raggedleft
Calculated Debris Mass (lb)\strut
\end{minipage} & \begin{minipage}[b]{0.13\columnwidth}\raggedleft
Ratio Calculated (Debris/Total)\strut
\end{minipage} & \begin{minipage}[b]{0.13\columnwidth}\raggedleft
Refinery Recorded Mass (lb)\strut
\end{minipage}\tabularnewline
\midrule
\endhead
\begin{minipage}[t]{0.17\columnwidth}\raggedright
Large End Cap\strut
\end{minipage} & \begin{minipage}[t]{0.09\columnwidth}\raggedleft
22.0\strut
\end{minipage} & \begin{minipage}[t]{0.09\columnwidth}\raggedleft
31541.8\strut
\end{minipage} & \begin{minipage}[t]{0.13\columnwidth}\raggedleft
0.56\strut
\end{minipage} & \begin{minipage}[t]{0.13\columnwidth}\raggedleft
41809.6\strut
\end{minipage}\tabularnewline
\begin{minipage}[t]{0.17\columnwidth}\raggedright
Small End Cap\strut
\end{minipage} & \begin{minipage}[t]{0.09\columnwidth}\raggedleft
5.0\strut
\end{minipage} & \begin{minipage}[t]{0.09\columnwidth}\raggedleft
10078.5\strut
\end{minipage} & \begin{minipage}[t]{0.13\columnwidth}\raggedleft
0.18\strut
\end{minipage} & \begin{minipage}[t]{0.13\columnwidth}\raggedleft
13438.8\strut
\end{minipage}\tabularnewline
\begin{minipage}[t]{0.17\columnwidth}\raggedright
Fillet\strut
\end{minipage} & \begin{minipage}[t]{0.09\columnwidth}\raggedleft
12.0\strut
\end{minipage} & \begin{minipage}[t]{0.09\columnwidth}\raggedleft
15150.5\strut
\end{minipage} & \begin{minipage}[t]{0.13\columnwidth}\raggedleft
0.26\strut
\end{minipage} & \begin{minipage}[t]{0.13\columnwidth}\raggedleft
19411.6\strut
\end{minipage}\tabularnewline
\begin{minipage}[t]{0.17\columnwidth}\raggedright
Total\strut
\end{minipage} & \begin{minipage}[t]{0.09\columnwidth}\raggedleft
39.0\strut
\end{minipage} & \begin{minipage}[t]{0.09\columnwidth}\raggedleft
56770.8\strut
\end{minipage} & \begin{minipage}[t]{0.13\columnwidth}\raggedleft
1.00\strut
\end{minipage} & \begin{minipage}[t]{0.13\columnwidth}\raggedleft
74660.0\strut
\end{minipage}\tabularnewline
\bottomrule
\end{longtable}

\hypertarget{boiling-liquid-expanding-vapor-explosion-bleve}{%
\subsection{Boiling-Liquid Expanding-Vapor Explosion
(BLEVE)}\label{boiling-liquid-expanding-vapor-explosion-bleve}}

A BLEVE results from the sudden failure of a tank containing a
compressed vapor (head space) and a super-heated liquid (a liquid heated
above it's boiling point but without boiling). The magnitude of the
blast depends on how super-heated the liquid was at failure. As the
level of super-heat rises, the portion of liquid that flash-boils rises,
thus increasing the energy released. Once containment failure occurs the
energy is distributed into four forms:

\begin{enumerate}
\def\labelenumi{\arabic{enumi}.}
\tightlist
\item
  Overpressure wave
\item
  Kinetic energy of fragments
\item
  Deformation and failure of the containment material
\item
  Heat transferred to environment
\end{enumerate}

The distribution of the energy into the these four forms depends on the
specifics of the explosion. Planas-Cuchi et al.~found that a
\emph{fragile} failure releases 80\% of the energy into the blastwave,
while a \emph{ductile} failure releases 40\% of the energy into the
blastwave. The remaining energy becomes kinetic energy of the fragments.
The heat transfer to the environment is relatively small
\cite{Planas2004}. In practice most pressure vessels are designed with
materials that are ductile rather than brittle to avoid sudden and
catastrophic brittle (fragile) failures \cite{Benac2016}.

\hypertarget{calculation-of-air-blast-from-a-bleve}{%
\section{Calculation of Air Blast from a
BLEVE}\label{calculation-of-air-blast-from-a-bleve}}

\hypertarget{pressure-at-state-1-pre-failure-state}{%
\subsection{Pressure at State 1 (Pre-failure
State)}\label{pressure-at-state-1-pre-failure-state}}

The drum (tank) is assumed to fail at \(1.21\) times the opening
pressure of the pressure relief valve (PRV)\cite{Engineers2013}. This
pressure is based on the American Petroleum Institutes Standard 521
which, requires that pressure relief valves on pressure vessels achieve
rated flow at 1.21 times the maximum allowable working pressure. The PRV
was set to \(155psig\:(1.07Mpa)\) therefore, the absolute pressure at
state 1 (failure state) is given by,

\begin{equation}p_1 = 1.21\left(p_{PRV}+p_{atm}\right)\end{equation}
\begin{equation}p_1 = 1.21\left(1068689.9+101325\right)\end{equation}
\begin{equation}p_1 = 169.7psi\:(1.17\:MPa)\end{equation}

\hypertarget{pressure-at-state-2-final-expanded-state}{%
\subsection{Pressure at State 2 (Final Expanded
State)}\label{pressure-at-state-2-final-expanded-state}}

The pressure at state 2 (final expanded state) is standard atmospheric
pressure or \(14.7psi\:(0.101MPa)\). The other state variables can be
determined based on the ``saturated'' state of the butane and butene
inside the tank and the known pressures. The thermodynamic data for
states 1 and 2 is summarized in Table (\cref{tbl:thermo}).

\begin{table}[H]
\caption{Propane Thermodynamic Data for Inital (1) and Final (2) States}\label{tbl:thermo}
\centering
\begin{adjustbox}{max width=\textwidth}
\begin{tabular}{lrrrrrrrrr}
\toprule
{} &  $P \left(kPa\right)$ &  $h_f \left(\frac{kJ}{kg}\right)$ &  $h_g \left(\frac{kJ}{kg}\right)$ &  $v_f \left(\frac{m^3}{kg}\right)$ &  $v_g \left(\frac{m^3}{kg}\right)$ &  $u_f \left(\frac{kJ}{kg\:K}\right)$ &  $u_g \left(\frac{kJ}{kg\:K}\right)$ &  $s_f \left(\frac{kJ}{kg\:K}\right)$ &  $s_g \left(\frac{kJ}{kg\:K}\right)$ \\
State    &                       &                                   &                                   &                                    &                                    &                                      &                                      &                                      &                                      \\
\midrule
Butane-1 &              1.42E+03 &                          4.51E+05 &                          6.15E+05 &                           2.15E-03 &                           3.20E-02 &                             3.08E+02 &                             5.70E+02 &                             1.37E+00 &                             2.34E+00 \\
Butane-2 &              1.01E+02 &                          1.99E+05 &                          5.26E+05 &                           1.72E-03 &                           4.14E-01 &                             1.00E+02 &                             4.84E+02 &                             6.07E-01 &                             2.45E+00 \\
Butene-1 &              1.42E+03 &                          2.29E+05 &                          5.02E+05 &                           2.02E-03 &                           2.83E-02 &                             3.08E+02 &                             5.70E+02 &                             7.24E-01 &                             1.48E+00 \\
Butene-2 &              1.01E+02 &                          8.76E-04 &                          3.92E+05 &                           1.60E-03 &                           3.74E-01 &                             1.00E+02 &                             4.84E+02 &                             3.71E-10 &                             1.47E+00 \\
\bottomrule
\end{tabular}

\end{adjustbox}
\end{table}

\hypertarget{internal-energy-at-states-1-and-2}{%
\section{Internal Energy at States 1 and
2}\label{internal-energy-at-states-1-and-2}}

\hypertarget{internal-energy-at-state-1}{%
\subsection{Internal Energy at State
1}\label{internal-energy-at-state-1}}

The internal energy at at state 1 (saturated) is calculated from,

\begin{equation}h = u + pv\end{equation}

where \(h\) is the enthalpy, \(p\) is the pressure, and \(v\) is the
specific volume. Therefore, for state 1 (fluid and gas) we have,

\begin{equation}u = h - pv\end{equation}

\begin{equation}u_{1f} = h_{1f} - p_1v_{1f}\end{equation}
\begin{equation}u_{1f} = 354584.4 - (1937.603\:kPa)(0.002287\:m^3/kg)\end{equation}
\begin{equation}u_{1f} = 350.15\:kJ/kg\end{equation}

\begin{equation}u_{1g} = h_{1g} - p_1v_{1g}\end{equation}
\begin{equation}u_{1g} = 100356.29 - (101.325\:kPa)(0.001722\:m^3/kg)\end{equation}
\begin{equation}u_{1g} = 581.68\:kJ/kg\end{equation}

and similarly for state 2,

\begin{equation}u_{2f} = 100.18\:kJ/kg\end{equation}
\begin{equation}u_{2g} = 484.01\:kJ/kg\end{equation}

\hypertarget{internal-energy-at-state-2}{%
\subsection{Internal Energy at State
2}\label{internal-energy-at-state-2}}

When the drum breaks and the propane at state 1 expands to state 2
(atmospheric pressure) some of the liquid propane vaporizes and some of
the gaseous propane condenses. Therefore unlike at the saturated state
1, there is both vapor and fluid present. We can calculate the vapor
present using the vapor quality \((\chi)\), from,

\begin{equation}\chi = \frac{\nu_{tot} - \nu_f}{\nu_g-\nu_f}\end{equation}

where \(\nu\) is the specific gravity. This equation is also true for
the entropy \((s)\), internal energy \((u)\), and enthalpy \((h)\).
Using the entropy \((s)\) we can calculate the quality of the saturated
liquid and vapor as the propane transitions from state 1 to state 2.
Therefore, the liquid vapor quality at state 2 is given by s

\begin{equation}\chi_f = \frac{s_{f1} - s_{f2}}{s_{g2}-s_{f2}}\end{equation}

\begin{equation}\chi_f = \frac{1.503188 - 0.607045}{2.449144-0.607045}\end{equation}

\begin{equation}\chi_f = 0.4865\end{equation}

and for the vapor at state 2,

\begin{equation}\chi_g = \frac{s_{g2} - s_{g1}}{s_{g2}-s_{f2}}\end{equation}

\begin{equation}\chi_g= \frac{2.449144 - 2.325970}{2.449144-0.607045}\end{equation}

\begin{equation}\chi_g = 0.06687\end{equation}

We can then calculate the internal energy at state 2 using,

\begin{equation}u_{2-fluid} = (1-\chi_f)u_{f2} + \chi_f u_{g2}\end{equation}
\begin{equation}u_{2-vapor} = (1-\chi_g)u_{g2} + \chi_g u_{f2}\end{equation}

\begin{equation}u_{2-fluid} = (1-0.4865)100.1818 + (0.4865)(484.0111)\end{equation}
\begin{equation}u_{2-vapor} = (1-0.0668)484.0111 + (0.0668)(100.1818)\end{equation}

\begin{equation}u_{2-fluid} = 286.9068\:kJ/kg\end{equation}
\begin{equation}u_{2-vapor} = 458.3459\:kJ/kg\end{equation}

\hypertarget{the-specific-work}{%
\section{The Specific Work}\label{the-specific-work}}

The work that the expanding vapor and fluid can perform is the
difference between the initial (1) and final (2) states,

\begin{equation}e_{ex} = u_1 - u_2\end{equation}

for the saturated fluid we have,

\begin{equation}e_{exf} = u_{f1} - u_{2-fluid}\end{equation}
\begin{equation}e_{exf} = 350.15 - 286.91\end{equation}
\begin{equation}e_{exf} = 63.25\:kJ/kg\end{equation}

and for the vapor,

\begin{equation}e_{exg} = u_{g1} - u_{2-vapor}\end{equation}
\begin{equation}e_{exg} = 581.68 - 458.35\end{equation}
\begin{equation}e_{exg} = 123.34\:kJ/kg\end{equation}

\begin{lstlisting}[language={},postbreak={},numbers=none,xrightmargin=7pt,belowskip=5pt,aboveskip=5pt,breakindent=0pt]
The mass of the fluid at state 1 is 34892.91 kg
The mass of the vapor at state 1 is  2056.33 kg
The explosion energy of the fluid at state 1 is 3394625.38 kJ.
The explosion energy of the vapor at state 1 is 448593.94 kJ.
The total energy of the surface explosion is Ex_tot = 3843219.32 kJ.

\end{lstlisting}

\hypertarget{the-explosion-energy}{%
\section{The Explosion Energy}\label{the-explosion-energy}}

The explosion energy is calcu

\begin{lstlisting}[language={},postbreak={},numbers=none,xrightmargin=7pt,belowskip=5pt,aboveskip=5pt,breakindent=0pt]
The non-dimensional range of the receptor is     0.44
The non-dimensional tank pressure p/po is    13.97

\end{lstlisting}

\begin{table}[H]
\centering
\begin{adjustbox}{max width=\textwidth}
\begin{tabular}{lrrrl}
\toprule
{} &        P &        R &        w &            z \\
\midrule
0 &     3.13 &     0.24 &     5.00 &  $p/p_0=5.0$ \\
1 &     3.12 &     0.25 &     5.00 &  $p/p_0=5.0$ \\
2 &     3.11 &     0.27 &     5.00 &  $p/p_0=5.0$ \\
3 &     3.09 &     0.29 &     5.00 &  $p/p_0=5.0$ \\
4 &     3.09 &     0.30 &     5.00 &  $p/p_0=5.0$ \\
\bottomrule
\end{tabular}

\end{adjustbox}
\end{table}

\hypertarget{nondimensional-side-on-pressure-and-impulse}{%
\subsection{Nondimensional Side-on Pressure and
Impulse}\label{nondimensional-side-on-pressure-and-impulse}}

The nondimenstional side-on pressure can be calculated from Figure X and
gives a \(\bar{P_s} = 0.064\) for an \(\bar{R}=4.4\) and a
\(p/p_0 = 19.12\). The nondimensional side-on impulse can be calculated
from Figure Y and gives a \(\bar{i_s} = 0.013\) for an \(\bar{R}=4.4\)
and a \(p/p_0 = 19.12\).

The side-on pressure and impulse can be calculated from the following:

\begin{equation}\bar{P_s} - p_0 = (0.064)(101.325\:kPa) = 6.5\:kPa\end{equation}
\begin{equation}i_s = \frac{(0.013)(101325\:kPa)^{2/3}(1160.9E6\:J)^{1/3}}{340\:m/s}= 86.31\:Pa {\text -} s\end{equation}

\bibliographystyle{unsrtnat}
\bibliography{refinery_bleve_files/library}

\end{document}
