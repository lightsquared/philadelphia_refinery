

% A latex document created by ipypublish
% outline: ipypublish.templates.outline_schemas/latex_outline.latex.j2
% with segments:
% - standard-standard_packages: with standard nbconvert packages
% - standard-standard_definitions: with standard nbconvert definitions
% - ipypublish-doc_article: with the main ipypublish article setup
% - ipypublish-front_pages: with the main ipypublish title and contents page setup
% - ipypublish-biblio_natbib: with the main ipypublish bibliography
% - ipypublish-contents_output: with the main ipypublish content
% - ipypublish-contents_framed_code: with the input code wrapped and framed
% - ipypublish-glossary: with the main ipypublish glossary
%
%%%%%%%%%%%% DOCCLASS

\documentclass[10pt,parskip=half,
toc=sectionentrywithdots,
bibliography=totocnumbered,
captions=tableheading,numbers=noendperiod]{scrartcl}
%\usepackage{polyglossia}
%\setmainlanguage{american}
%\DeclareTextCommandDefault{\nobreakspace}{\leavevmode\nobreak\ }
\usepackage[american]{babel}

%%%%%%%%%%%%

%%%%%%%%%%%% PACKAGES

\usepackage[T1]{fontenc} % Nicer default font (+ math font) than Computer Modern for most use cases
\usepackage{mathpazo}
\usepackage{graphicx}
\usepackage[skip=3pt]{caption}
\usepackage{adjustbox} % Used to constrain images to a maximum size
\usepackage[table]{xcolor} % Allow colors to be defined
\usepackage{enumerate} % Needed for markdown enumerations to work
\usepackage{amsmath} % Equations
\usepackage{amssymb} % Equations
\usepackage{textcomp} % defines textquotesingle
% Hack from http://tex.stackexchange.com/a/47451/13684:
\AtBeginDocument{%
    \def\PYZsq{\textquotesingle}% Upright quotes in Pygmentized code
}
\usepackage{upquote} % Upright quotes for verbatim code
\usepackage{eurosym} % defines \euro
\usepackage[mathletters]{ucs} % Extended unicode (utf-8) support
\usepackage[utf8x]{inputenc} % Allow utf-8 characters in the tex document
\usepackage{fancyvrb} % verbatim replacement that allows latex
\usepackage{grffile} % extends the file name processing of package graphics
                        % to support a larger range
% The hyperref package gives us a pdf with properly built
% internal navigation ('pdf bookmarks' for the table of contents,
% internal cross-reference links, web links for URLs, etc.)
\usepackage{hyperref}
\usepackage{longtable} % longtable support required by pandoc >1.10
\usepackage{booktabs}  % table support for pandoc > 1.12.2
\usepackage[inline]{enumitem} % IRkernel/repr support (it uses the enumerate* environment)
\usepackage[normalem]{ulem} % ulem is needed to support strikethroughs (\sout)
                            % normalem makes italics be italics, not underlines

\usepackage{translations}
\usepackage{microtype} % improves the spacing between words and letters
\usepackage{placeins} % placement of figures
% could use \usepackage[section]{placeins} but placing in subsection in command section
% Places the float at precisely the location in the LaTeX code (with H)
\usepackage{float}
\usepackage[colorinlistoftodos,obeyFinal,textwidth=.8in]{todonotes} % to mark to-dos
% number figures, tables and equations by section
% fix for new versions of texlive (see https://tex.stackexchange.com/a/425603/107738)
\let\counterwithout\relax
\let\counterwithin\relax
\usepackage{chngcntr}
% header/footer
\usepackage[footsepline=0.25pt]{scrlayer-scrpage}

% bibliography formatting
\usepackage[numbers, square, super, sort&compress]{natbib}
% hyperlink doi's
\usepackage{doi}

    % define a code float
    \usepackage{newfloat} % to define a new float types
    \DeclareFloatingEnvironment[
        fileext=frm,placement={!ht},
        within=section,name=Code]{codecell}
    \DeclareFloatingEnvironment[
        fileext=frm,placement={!ht},
        within=section,name=Text]{textcell}
    \DeclareFloatingEnvironment[
        fileext=frm,placement={!ht},
        within=section,name=Text]{errorcell}

    \usepackage{listings} % a package for wrapping code in a box
    \usepackage[framemethod=tikz]{mdframed} % to fram code

%%%%%%%%%%%%

%%%%%%%%%%%% DEFINITIONS

% Pygments definitions

\makeatletter
\def\PY@reset{\let\PY@it=\relax \let\PY@bf=\relax%
    \let\PY@ul=\relax \let\PY@tc=\relax%
    \let\PY@bc=\relax \let\PY@ff=\relax}
\def\PY@tok#1{\csname PY@tok@#1\endcsname}
\def\PY@toks#1+{\ifx\relax#1\empty\else%
    \PY@tok{#1}\expandafter\PY@toks\fi}
\def\PY@do#1{\PY@bc{\PY@tc{\PY@ul{%
    \PY@it{\PY@bf{\PY@ff{#1}}}}}}}
\def\PY#1#2{\PY@reset\PY@toks#1+\relax+\PY@do{#2}}

\expandafter\def\csname PY@tok@w\endcsname{\def\PY@tc##1{\textcolor[rgb]{0.73,0.73,0.73}{##1}}}
\expandafter\def\csname PY@tok@c\endcsname{\let\PY@it=\textit\def\PY@tc##1{\textcolor[rgb]{0.25,0.50,0.50}{##1}}}
\expandafter\def\csname PY@tok@cp\endcsname{\def\PY@tc##1{\textcolor[rgb]{0.74,0.48,0.00}{##1}}}
\expandafter\def\csname PY@tok@k\endcsname{\let\PY@bf=\textbf\def\PY@tc##1{\textcolor[rgb]{0.00,0.50,0.00}{##1}}}
\expandafter\def\csname PY@tok@kp\endcsname{\def\PY@tc##1{\textcolor[rgb]{0.00,0.50,0.00}{##1}}}
\expandafter\def\csname PY@tok@kt\endcsname{\def\PY@tc##1{\textcolor[rgb]{0.69,0.00,0.25}{##1}}}
\expandafter\def\csname PY@tok@o\endcsname{\def\PY@tc##1{\textcolor[rgb]{0.40,0.40,0.40}{##1}}}
\expandafter\def\csname PY@tok@ow\endcsname{\let\PY@bf=\textbf\def\PY@tc##1{\textcolor[rgb]{0.67,0.13,1.00}{##1}}}
\expandafter\def\csname PY@tok@nb\endcsname{\def\PY@tc##1{\textcolor[rgb]{0.00,0.50,0.00}{##1}}}
\expandafter\def\csname PY@tok@nf\endcsname{\def\PY@tc##1{\textcolor[rgb]{0.00,0.00,1.00}{##1}}}
\expandafter\def\csname PY@tok@nc\endcsname{\let\PY@bf=\textbf\def\PY@tc##1{\textcolor[rgb]{0.00,0.00,1.00}{##1}}}
\expandafter\def\csname PY@tok@nn\endcsname{\let\PY@bf=\textbf\def\PY@tc##1{\textcolor[rgb]{0.00,0.00,1.00}{##1}}}
\expandafter\def\csname PY@tok@ne\endcsname{\let\PY@bf=\textbf\def\PY@tc##1{\textcolor[rgb]{0.82,0.25,0.23}{##1}}}
\expandafter\def\csname PY@tok@nv\endcsname{\def\PY@tc##1{\textcolor[rgb]{0.10,0.09,0.49}{##1}}}
\expandafter\def\csname PY@tok@no\endcsname{\def\PY@tc##1{\textcolor[rgb]{0.53,0.00,0.00}{##1}}}
\expandafter\def\csname PY@tok@nl\endcsname{\def\PY@tc##1{\textcolor[rgb]{0.63,0.63,0.00}{##1}}}
\expandafter\def\csname PY@tok@ni\endcsname{\let\PY@bf=\textbf\def\PY@tc##1{\textcolor[rgb]{0.60,0.60,0.60}{##1}}}
\expandafter\def\csname PY@tok@na\endcsname{\def\PY@tc##1{\textcolor[rgb]{0.49,0.56,0.16}{##1}}}
\expandafter\def\csname PY@tok@nt\endcsname{\let\PY@bf=\textbf\def\PY@tc##1{\textcolor[rgb]{0.00,0.50,0.00}{##1}}}
\expandafter\def\csname PY@tok@nd\endcsname{\def\PY@tc##1{\textcolor[rgb]{0.67,0.13,1.00}{##1}}}
\expandafter\def\csname PY@tok@s\endcsname{\def\PY@tc##1{\textcolor[rgb]{0.73,0.13,0.13}{##1}}}
\expandafter\def\csname PY@tok@sd\endcsname{\let\PY@it=\textit\def\PY@tc##1{\textcolor[rgb]{0.73,0.13,0.13}{##1}}}
\expandafter\def\csname PY@tok@si\endcsname{\let\PY@bf=\textbf\def\PY@tc##1{\textcolor[rgb]{0.73,0.40,0.53}{##1}}}
\expandafter\def\csname PY@tok@se\endcsname{\let\PY@bf=\textbf\def\PY@tc##1{\textcolor[rgb]{0.73,0.40,0.13}{##1}}}
\expandafter\def\csname PY@tok@sr\endcsname{\def\PY@tc##1{\textcolor[rgb]{0.73,0.40,0.53}{##1}}}
\expandafter\def\csname PY@tok@ss\endcsname{\def\PY@tc##1{\textcolor[rgb]{0.10,0.09,0.49}{##1}}}
\expandafter\def\csname PY@tok@sx\endcsname{\def\PY@tc##1{\textcolor[rgb]{0.00,0.50,0.00}{##1}}}
\expandafter\def\csname PY@tok@m\endcsname{\def\PY@tc##1{\textcolor[rgb]{0.40,0.40,0.40}{##1}}}
\expandafter\def\csname PY@tok@gh\endcsname{\let\PY@bf=\textbf\def\PY@tc##1{\textcolor[rgb]{0.00,0.00,0.50}{##1}}}
\expandafter\def\csname PY@tok@gu\endcsname{\let\PY@bf=\textbf\def\PY@tc##1{\textcolor[rgb]{0.50,0.00,0.50}{##1}}}
\expandafter\def\csname PY@tok@gd\endcsname{\def\PY@tc##1{\textcolor[rgb]{0.63,0.00,0.00}{##1}}}
\expandafter\def\csname PY@tok@gi\endcsname{\def\PY@tc##1{\textcolor[rgb]{0.00,0.63,0.00}{##1}}}
\expandafter\def\csname PY@tok@gr\endcsname{\def\PY@tc##1{\textcolor[rgb]{1.00,0.00,0.00}{##1}}}
\expandafter\def\csname PY@tok@ge\endcsname{\let\PY@it=\textit}
\expandafter\def\csname PY@tok@gs\endcsname{\let\PY@bf=\textbf}
\expandafter\def\csname PY@tok@gp\endcsname{\let\PY@bf=\textbf\def\PY@tc##1{\textcolor[rgb]{0.00,0.00,0.50}{##1}}}
\expandafter\def\csname PY@tok@go\endcsname{\def\PY@tc##1{\textcolor[rgb]{0.53,0.53,0.53}{##1}}}
\expandafter\def\csname PY@tok@gt\endcsname{\def\PY@tc##1{\textcolor[rgb]{0.00,0.27,0.87}{##1}}}
\expandafter\def\csname PY@tok@err\endcsname{\def\PY@bc##1{\setlength{\fboxsep}{0pt}\fcolorbox[rgb]{1.00,0.00,0.00}{1,1,1}{\strut ##1}}}
\expandafter\def\csname PY@tok@kc\endcsname{\let\PY@bf=\textbf\def\PY@tc##1{\textcolor[rgb]{0.00,0.50,0.00}{##1}}}
\expandafter\def\csname PY@tok@kd\endcsname{\let\PY@bf=\textbf\def\PY@tc##1{\textcolor[rgb]{0.00,0.50,0.00}{##1}}}
\expandafter\def\csname PY@tok@kn\endcsname{\let\PY@bf=\textbf\def\PY@tc##1{\textcolor[rgb]{0.00,0.50,0.00}{##1}}}
\expandafter\def\csname PY@tok@kr\endcsname{\let\PY@bf=\textbf\def\PY@tc##1{\textcolor[rgb]{0.00,0.50,0.00}{##1}}}
\expandafter\def\csname PY@tok@bp\endcsname{\def\PY@tc##1{\textcolor[rgb]{0.00,0.50,0.00}{##1}}}
\expandafter\def\csname PY@tok@fm\endcsname{\def\PY@tc##1{\textcolor[rgb]{0.00,0.00,1.00}{##1}}}
\expandafter\def\csname PY@tok@vc\endcsname{\def\PY@tc##1{\textcolor[rgb]{0.10,0.09,0.49}{##1}}}
\expandafter\def\csname PY@tok@vg\endcsname{\def\PY@tc##1{\textcolor[rgb]{0.10,0.09,0.49}{##1}}}
\expandafter\def\csname PY@tok@vi\endcsname{\def\PY@tc##1{\textcolor[rgb]{0.10,0.09,0.49}{##1}}}
\expandafter\def\csname PY@tok@vm\endcsname{\def\PY@tc##1{\textcolor[rgb]{0.10,0.09,0.49}{##1}}}
\expandafter\def\csname PY@tok@sa\endcsname{\def\PY@tc##1{\textcolor[rgb]{0.73,0.13,0.13}{##1}}}
\expandafter\def\csname PY@tok@sb\endcsname{\def\PY@tc##1{\textcolor[rgb]{0.73,0.13,0.13}{##1}}}
\expandafter\def\csname PY@tok@sc\endcsname{\def\PY@tc##1{\textcolor[rgb]{0.73,0.13,0.13}{##1}}}
\expandafter\def\csname PY@tok@dl\endcsname{\def\PY@tc##1{\textcolor[rgb]{0.73,0.13,0.13}{##1}}}
\expandafter\def\csname PY@tok@s2\endcsname{\def\PY@tc##1{\textcolor[rgb]{0.73,0.13,0.13}{##1}}}
\expandafter\def\csname PY@tok@sh\endcsname{\def\PY@tc##1{\textcolor[rgb]{0.73,0.13,0.13}{##1}}}
\expandafter\def\csname PY@tok@s1\endcsname{\def\PY@tc##1{\textcolor[rgb]{0.73,0.13,0.13}{##1}}}
\expandafter\def\csname PY@tok@mb\endcsname{\def\PY@tc##1{\textcolor[rgb]{0.40,0.40,0.40}{##1}}}
\expandafter\def\csname PY@tok@mf\endcsname{\def\PY@tc##1{\textcolor[rgb]{0.40,0.40,0.40}{##1}}}
\expandafter\def\csname PY@tok@mh\endcsname{\def\PY@tc##1{\textcolor[rgb]{0.40,0.40,0.40}{##1}}}
\expandafter\def\csname PY@tok@mi\endcsname{\def\PY@tc##1{\textcolor[rgb]{0.40,0.40,0.40}{##1}}}
\expandafter\def\csname PY@tok@il\endcsname{\def\PY@tc##1{\textcolor[rgb]{0.40,0.40,0.40}{##1}}}
\expandafter\def\csname PY@tok@mo\endcsname{\def\PY@tc##1{\textcolor[rgb]{0.40,0.40,0.40}{##1}}}
\expandafter\def\csname PY@tok@ch\endcsname{\let\PY@it=\textit\def\PY@tc##1{\textcolor[rgb]{0.25,0.50,0.50}{##1}}}
\expandafter\def\csname PY@tok@cm\endcsname{\let\PY@it=\textit\def\PY@tc##1{\textcolor[rgb]{0.25,0.50,0.50}{##1}}}
\expandafter\def\csname PY@tok@cpf\endcsname{\let\PY@it=\textit\def\PY@tc##1{\textcolor[rgb]{0.25,0.50,0.50}{##1}}}
\expandafter\def\csname PY@tok@c1\endcsname{\let\PY@it=\textit\def\PY@tc##1{\textcolor[rgb]{0.25,0.50,0.50}{##1}}}
\expandafter\def\csname PY@tok@cs\endcsname{\let\PY@it=\textit\def\PY@tc##1{\textcolor[rgb]{0.25,0.50,0.50}{##1}}}

\def\PYZbs{\char`\\}
\def\PYZus{\char`\_}
\def\PYZob{\char`\{}
\def\PYZcb{\char`\}}
\def\PYZca{\char`\^}
\def\PYZam{\char`\&}
\def\PYZlt{\char`\<}
\def\PYZgt{\char`\>}
\def\PYZsh{\char`\#}
\def\PYZpc{\char`\%}
\def\PYZdl{\char`\$}
\def\PYZhy{\char`\-}
\def\PYZsq{\char`\'}
\def\PYZdq{\char`\"}
\def\PYZti{\char`\~}
% for compatibility with earlier versions
\def\PYZat{@}
\def\PYZlb{[}
\def\PYZrb{]}
\makeatother

% ANSI colors
\definecolor{ansi-black}{HTML}{3E424D}
\definecolor{ansi-black-intense}{HTML}{282C36}
\definecolor{ansi-red}{HTML}{E75C58}
\definecolor{ansi-red-intense}{HTML}{B22B31}
\definecolor{ansi-green}{HTML}{00A250}
\definecolor{ansi-green-intense}{HTML}{007427}
\definecolor{ansi-yellow}{HTML}{DDB62B}
\definecolor{ansi-yellow-intense}{HTML}{B27D12}
\definecolor{ansi-blue}{HTML}{208FFB}
\definecolor{ansi-blue-intense}{HTML}{0065CA}
\definecolor{ansi-magenta}{HTML}{D160C4}
\definecolor{ansi-magenta-intense}{HTML}{A03196}
\definecolor{ansi-cyan}{HTML}{60C6C8}
\definecolor{ansi-cyan-intense}{HTML}{258F8F}
\definecolor{ansi-white}{HTML}{C5C1B4}
\definecolor{ansi-white-intense}{HTML}{A1A6B2}

% commands and environments needed by pandoc snippets
% extracted from the output of `pandoc -s`
\providecommand{\tightlist}{%
  \setlength{\itemsep}{0pt}\setlength{\parskip}{0pt}}
\DefineVerbatimEnvironment{Highlighting}{Verbatim}{commandchars=\\\{\}}
% Add ',fontsize=\small' for more characters per line
\newenvironment{Shaded}{}{}
\newcommand{\KeywordTok}[1]{\textcolor[rgb]{0.00,0.44,0.13}{\textbf{{#1}}}}
\newcommand{\DataTypeTok}[1]{\textcolor[rgb]{0.56,0.13,0.00}{{#1}}}
\newcommand{\DecValTok}[1]{\textcolor[rgb]{0.25,0.63,0.44}{{#1}}}
\newcommand{\BaseNTok}[1]{\textcolor[rgb]{0.25,0.63,0.44}{{#1}}}
\newcommand{\FloatTok}[1]{\textcolor[rgb]{0.25,0.63,0.44}{{#1}}}
\newcommand{\CharTok}[1]{\textcolor[rgb]{0.25,0.44,0.63}{{#1}}}
\newcommand{\StringTok}[1]{\textcolor[rgb]{0.25,0.44,0.63}{{#1}}}
\newcommand{\CommentTok}[1]{\textcolor[rgb]{0.38,0.63,0.69}{\textit{{#1}}}}
\newcommand{\OtherTok}[1]{\textcolor[rgb]{0.00,0.44,0.13}{{#1}}}
\newcommand{\AlertTok}[1]{\textcolor[rgb]{1.00,0.00,0.00}{\textbf{{#1}}}}
\newcommand{\FunctionTok}[1]{\textcolor[rgb]{0.02,0.16,0.49}{{#1}}}
\newcommand{\RegionMarkerTok}[1]{{#1}}
\newcommand{\ErrorTok}[1]{\textcolor[rgb]{1.00,0.00,0.00}{\textbf{{#1}}}}
\newcommand{\NormalTok}[1]{{#1}}

% Additional commands for more recent versions of Pandoc
\newcommand{\ConstantTok}[1]{\textcolor[rgb]{0.53,0.00,0.00}{{#1}}}
\newcommand{\SpecialCharTok}[1]{\textcolor[rgb]{0.25,0.44,0.63}{{#1}}}
\newcommand{\VerbatimStringTok}[1]{\textcolor[rgb]{0.25,0.44,0.63}{{#1}}}
\newcommand{\SpecialStringTok}[1]{\textcolor[rgb]{0.73,0.40,0.53}{{#1}}}
\newcommand{\ImportTok}[1]{{#1}}
\newcommand{\DocumentationTok}[1]{\textcolor[rgb]{0.73,0.13,0.13}{\textit{{#1}}}}
\newcommand{\AnnotationTok}[1]{\textcolor[rgb]{0.38,0.63,0.69}{\textbf{\textit{{#1}}}}}
\newcommand{\CommentVarTok}[1]{\textcolor[rgb]{0.38,0.63,0.69}{\textbf{\textit{{#1}}}}}
\newcommand{\VariableTok}[1]{\textcolor[rgb]{0.10,0.09,0.49}{{#1}}}
\newcommand{\ControlFlowTok}[1]{\textcolor[rgb]{0.00,0.44,0.13}{\textbf{{#1}}}}
\newcommand{\OperatorTok}[1]{\textcolor[rgb]{0.40,0.40,0.40}{{#1}}}
\newcommand{\BuiltInTok}[1]{{#1}}
\newcommand{\ExtensionTok}[1]{{#1}}
\newcommand{\PreprocessorTok}[1]{\textcolor[rgb]{0.74,0.48,0.00}{{#1}}}
\newcommand{\AttributeTok}[1]{\textcolor[rgb]{0.49,0.56,0.16}{{#1}}}
\newcommand{\InformationTok}[1]{\textcolor[rgb]{0.38,0.63,0.69}{\textbf{\textit{{#1}}}}}
\newcommand{\WarningTok}[1]{\textcolor[rgb]{0.38,0.63,0.69}{\textbf{\textit{{#1}}}}}

% Define a nice break command that doesn't care if a line doesn't already
% exist.
\def\br{\hspace*{\fill} \\* }

% Math Jax compatability definitions
\def\gt{>}
\def\lt{<}

\setcounter{secnumdepth}{5}

% Colors for the hyperref package
\definecolor{urlcolor}{rgb}{0,.145,.698}
\definecolor{linkcolor}{rgb}{.71,0.21,0.01}
\definecolor{citecolor}{rgb}{.12,.54,.11}

\DeclareTranslationFallback{Author}{Author}
\DeclareTranslation{Portuges}{Author}{Autor}

\DeclareTranslationFallback{List of Codes}{List of Codes}
\DeclareTranslation{Catalan}{List of Codes}{Llista de Codis}
\DeclareTranslation{Danish}{List of Codes}{Liste over Koder}
\DeclareTranslation{German}{List of Codes}{Liste der Codes}
\DeclareTranslation{Spanish}{List of Codes}{Lista de C\'{o}digos}
\DeclareTranslation{French}{List of Codes}{Liste des Codes}
\DeclareTranslation{Italian}{List of Codes}{Elenco dei Codici}
\DeclareTranslation{Dutch}{List of Codes}{Lijst van Codes}
\DeclareTranslation{Portuges}{List of Codes}{Lista de C\'{o}digos}

\DeclareTranslationFallback{Supervisors}{Supervisors}
\DeclareTranslation{Catalan}{Supervisors}{Supervisors}
\DeclareTranslation{Danish}{Supervisors}{Vejledere}
\DeclareTranslation{German}{Supervisors}{Vorgesetzten}
\DeclareTranslation{Spanish}{Supervisors}{Supervisores}
\DeclareTranslation{French}{Supervisors}{Superviseurs}
\DeclareTranslation{Italian}{Supervisors}{Le autorit\`{a} di vigilanza}
\DeclareTranslation{Dutch}{Supervisors}{supervisors}
\DeclareTranslation{Portuguese}{Supervisors}{Supervisores}

\definecolor{codegreen}{rgb}{0,0.6,0}
\definecolor{codegray}{rgb}{0.5,0.5,0.5}
\definecolor{codepurple}{rgb}{0.58,0,0.82}
\definecolor{backcolour}{rgb}{0.95,0.95,0.95}

\lstdefinestyle{mystyle}{
    commentstyle=\color{codegreen},
    keywordstyle=\color{magenta},
    numberstyle=\tiny\color{codegray},
    stringstyle=\color{codepurple},
    basicstyle=\ttfamily,
    breakatwhitespace=false,
    keepspaces=true,
    numbers=left,
    numbersep=10pt,
    showspaces=false,
    showstringspaces=false,
    showtabs=false,
    tabsize=2,
    breaklines=true,
    literate={\-}{}{0\discretionary{-}{}{-}},
  postbreak=\mbox{\textcolor{red}{$\hookrightarrow$}\space},
}

\lstset{style=mystyle}

\surroundwithmdframed[
  hidealllines=true,
  backgroundcolor=backcolour,
  innerleftmargin=0pt,
  innerrightmargin=0pt,
  innertopmargin=0pt,
  innerbottommargin=0pt]{lstlisting}

%%%%%%%%%%%%

%%%%%%%%%%%% MARGINS

 % Used to adjust the document margins
\usepackage{geometry}
\geometry{tmargin=1in,bmargin=1in,lmargin=1in,rmargin=1in,
nohead,includefoot,footskip=25pt}
% you can use showframe option to check the margins visually
%%%%%%%%%%%%

%%%%%%%%%%%% COMMANDS

% ensure new section starts on new page
\addtokomafont{section}{\clearpage}

% Prevent overflowing lines due to hard-to-break entities
\sloppy

% Setup hyperref package
\hypersetup{
    breaklinks=true,  % so long urls are correctly broken across lines
    colorlinks=true,
    urlcolor=urlcolor,
    linkcolor=linkcolor,
    citecolor=citecolor,
    }

% ensure figures are placed within subsections
\makeatletter
\AtBeginDocument{%
    \expandafter\renewcommand\expandafter\subsection\expandafter
    {\expandafter\@fb@secFB\subsection}%
    \newcommand\@fb@secFB{\FloatBarrier
    \gdef\@fb@afterHHook{\@fb@topbarrier \gdef\@fb@afterHHook{}}}%
    \g@addto@macro\@afterheading{\@fb@afterHHook}%
    \gdef\@fb@afterHHook{}%
}
\makeatother

% number figures, tables and equations by section
\counterwithout{figure}{section}
\counterwithout{table}{section}
\counterwithout{equation}{section}
\makeatletter
\@addtoreset{table}{section}
\@addtoreset{figure}{section}
\@addtoreset{equation}{section}
\makeatother
\renewcommand\thetable{\thesection.\arabic{table}}
\renewcommand\thefigure{\thesection.\arabic{figure}}
\renewcommand\theequation{\thesection.\arabic{equation}}

    % set global options for float placement
    \makeatletter
        \providecommand*\setfloatlocations[2]{\@namedef{fps@#1}{#2}}
    \makeatother

% align captions to left (indented)
\captionsetup{justification=raggedright,
singlelinecheck=false,format=hang,labelfont={it,bf}}

% shift footer down so space between separation line
\ModifyLayer[addvoffset=.6ex]{scrheadings.foot.odd}
\ModifyLayer[addvoffset=.6ex]{scrheadings.foot.even}
\ModifyLayer[addvoffset=.6ex]{scrheadings.foot.oneside}
\ModifyLayer[addvoffset=.6ex]{plain.scrheadings.foot.odd}
\ModifyLayer[addvoffset=.6ex]{plain.scrheadings.foot.even}
\ModifyLayer[addvoffset=.6ex]{plain.scrheadings.foot.oneside}
\pagestyle{scrheadings}
\clearscrheadfoot{}
\ifoot{\leftmark}
\renewcommand{\sectionmark}[1]{\markleft{\thesection\ #1}}
\ofoot{\pagemark}
\cfoot{}

%%%%%%%%%%%%

%%%%%%%%%%%% FINAL HEADER MATERIAL

% clereref must be loaded after anything that changes the referencing system
\usepackage{cleveref}
\creflabelformat{equation}{#2#1#3}

% make the code float work with cleverref
\crefname{codecell}{code}{codes}
\Crefname{codecell}{code}{codes}
% make the text float work with cleverref
\crefname{textcell}{text}{texts}
\Crefname{textcell}{text}{texts}
% make the text float work with cleverref
\crefname{errorcell}{error}{errors}
\Crefname{errorcell}{error}{errors}

%%%%%%%%%%%%

\begin{document}

    \begin{titlepage}
  \begin{flushright}
    \includegraphics[width=0.7\textwidth]{philadelphia_refinery_files/fig_NCETR.png}
  \end{flushright}

  \begin{center}

  \vspace*{1cm}

  \Huge\textbf{Analysis of the Fire and Explosion at the Girard Point Refinery}

  \vspace{0.5cm}\LARGE{Boiling Liquid Expanding Vapor Explosion}

  \vspace{1.5cm}

  \begin{minipage}{0.8\textwidth}
    \begin{center}
    \begin{minipage}{0.39\textwidth}
    \begin{flushleft} \Large
    \emph{\GetTranslation{Author}:}\\S. Kevin McNeill, SA, ATF\\\href{mailto:shonn.mcneill@atf.gov}{shonn.mcneill@atf.gov}
    \end{flushleft}
    \end{minipage}
    \hspace{\fill}
    \begin{minipage}{0.39\textwidth}
    \begin{flushright} \Large
    \end{flushright}
    \end{minipage}
    \end{center}
  \end{minipage}

  \vfill

  \begin{minipage}{0.8\textwidth}
  \begin{center}\LARGE{Scientia est Potentia.}
  \end{center}
  \end{minipage}

  \vspace{0.8cm}
      \LARGE{National Center for Explosives Training and Research}\\
      \LARGE{Explosives Research and Development Division}\\

  \vspace{0.4cm}

  \today

  \end{center}
  \end{titlepage}

    \begingroup
    \let\cleardoublepage\relax
    \let\clearpage\relax\tableofcontents\listoffigures\listoftables
    \endgroup

\hypertarget{introduction}{%
\section{Introduction}\label{introduction}}

A fire and subsequent explosion occurred at approximately 4:22am on June
21, 2019, at the Philadelphia Energy Solutions Girard Point Refinery,
see \cref{fig:fire} \cite{Renshaw2019}. The explosion took place in the
V1 treater-feed-surge-drum (TFSD) within the pretreatment unit. The TFSD
explosion propelled a \(41809.6lb\:(18964.5kg)\) piece of the steel drum
approximately \(2100ft\:(640m)\) from the blast seat, see
\cref{fig:tank}. It is hypothesized that a boiling-liquid
expanding-vapor explosion (BLEVE) event generated the blast wave and
broke the drum into fragments. The Philadelphia Fire Department
requested ATF estimate the blast overpressure, debris throw, and thermal
effects generated when the tank exploded. This paper will estimate these
blast parameters assuming a BLEVE occurred. The analysis is based upon
an adiabatic and isentropic energy analysis developed by the Center for
Chemical Process Safety \cite{Safety2010}. Analysis was completed using
Jupyter Notebook running Python 3.7 and published using
ipypublish\cite{Sewell2019}\cite{Kluyver2019}.

\begin{figure}[H]
\hypertarget{fig:fire}{%
\begin{center}
\adjustimage{max size={0.9\linewidth}{0.9\paperheight},width=1.0\linewidth}{philadelphia_refinery_files/fig_refinery_fire.png}
\end{center}
\caption{A view of the Philadelphia Energy Solutions Incs oil refinery on fire
June 21, 2019 \cite{Maykuth2019}.}\label{fig:fire}
}
\end{figure}

    \begin{figure}[H]\begin{center}\adjustimage{max size={0.9\linewidth}{0.9\paperheight}}{philadelphia_refinery_files/output_3_0.png}\end{center}\caption{Largest piece of the TFSD was propelled \(2100ft (640m)\) from the blast
seat. The upper image shows the drum fragment in flight moving left to
right. The second image shows the drum fragment after striking the edge
of the Schuylkill River shoreline.}\label{fig:tank}\end{figure}

\hypertarget{background}{%
\section{Background}\label{background}}

\hypertarget{refinery}{%
\subsection{Refinery}\label{refinery}}

The Girard Point Refinery is the largest refinery on the East Coast,
located in southwest Philadelphia, PA, on the Schuylkill River, see
\cref{fig:map}. Prior to the fire and explosion, the refinery produced
approximately 335,000 barrels of gasoline per day
\cite{AssociatedPress2019}.

\begin{figure}[H]
\hypertarget{fig:map}{%
\begin{center}
\adjustimage{max size={0.9\linewidth}{0.9\paperheight},width=1.0\linewidth}{philadelphia_refinery_files/fig_map_girard.jpeg}
\end{center}
\caption{Map of Girard Point Refinery showing the blast seat location
\cite{Duchneskie2019}.}\label{fig:map}
}
\end{figure}

\hypertarget{treater-feed-surge-drum}{%
\subsection{Treater Feed Surge Drum}\label{treater-feed-surge-drum}}

The TFSD, located between the fluid catalytic cracker and the alkylation
unit, was part of the alkylation pretreatment process, see
\cref{fig:fig_pf}. The purpose of a surge drum is to stabilize
fluctuations in the overall system flow rate. During pretreatment, also
referred to as sweetening, sulfur compounds (hydrogen sulfide, thiophene
and mercaptan) are removed to improve color, odor, and oxidation
stability. Following pretreatment, alkylation generally converts
propylene \((C_3H_6)\), butylene \((C_4H_8)\), pentene \((C_5H_{10})\),
and isobutane \((C_3H_{10})\) to alkane liquids such as isoheptane
\((C_7H_{16})\) and isooctane \((C_8H_{16})\). Because of their high
octane and low vapor pressure, these alkylates are a highly valued
component in the production of gasoline \cite{flowserve2000}.

\begin{figure}[H]
\hypertarget{fig:fig_pf}{%
\begin{center}
\adjustimage{max size={0.9\linewidth}{0.9\paperheight},width=1.0\linewidth}{philadelphia_refinery_files/fig_process_flow.png}
\end{center}
\caption{Simple flow diagram showing the TFSD in the alkylation pretreatment
process\cite{Temur2014,Malone2019}.}\label{fig:fig_pf}
}
\end{figure}

The TFSD tank measured \(39.66ft\:(12.09m)\) in length, not including
the heads, see \cref{tbl:tbl_tank} and \cref{fig:fig_tank} for
construction details \cite{PES2019}. At the time of the explosion, the
TFSD contained \(20160gal\:(76.3m^3)\) of butane (50\% by volume) and
butene (40\% by volume) and other lesser constituients, see
\cref{tbl:tbl_chemicals_in_tank} for a complete list of chemicals
present.

\begin{longtable}[]{@{}rll@{}}
\caption{Treater-Feed-Surge-Drum Construction Parameters \cite{PES2019}
\label{tbl:tbl_tank}}\tabularnewline
\toprule
\begin{minipage}[b]{0.23\columnwidth}\raggedleft
Parameter\strut
\end{minipage} & \begin{minipage}[b]{0.23\columnwidth}\raggedright
Value\strut
\end{minipage} & \begin{minipage}[b]{0.23\columnwidth}\raggedright
Units\strut
\end{minipage}\tabularnewline
\midrule
\endfirsthead
\toprule
\begin{minipage}[b]{0.23\columnwidth}\raggedleft
Parameter\strut
\end{minipage} & \begin{minipage}[b]{0.23\columnwidth}\raggedright
Value\strut
\end{minipage} & \begin{minipage}[b]{0.23\columnwidth}\raggedright
Units\strut
\end{minipage}\tabularnewline
\midrule
\endhead
\begin{minipage}[t]{0.23\columnwidth}\raggedleft
Type\strut
\end{minipage} & \begin{minipage}[t]{0.23\columnwidth}\raggedright
horizontal\strut
\end{minipage} & \begin{minipage}[t]{0.23\columnwidth}\raggedright
NA\strut
\end{minipage}\tabularnewline
\begin{minipage}[t]{0.23\columnwidth}\raggedleft
Year Built\strut
\end{minipage} & \begin{minipage}[t]{0.23\columnwidth}\raggedright
1972\strut
\end{minipage} & \begin{minipage}[t]{0.23\columnwidth}\raggedright
NA\strut
\end{minipage}\tabularnewline
\begin{minipage}[t]{0.23\columnwidth}\raggedleft
Construction Material\strut
\end{minipage} & \begin{minipage}[t]{0.23\columnwidth}\raggedright
A516 Type 70 Steel\strut
\end{minipage} & \begin{minipage}[t]{0.23\columnwidth}\raggedright
NA\strut
\end{minipage}\tabularnewline
\begin{minipage}[t]{0.23\columnwidth}\raggedleft
Tank Wall Thickness\strut
\end{minipage} & \begin{minipage}[t]{0.23\columnwidth}\raggedright
0.8125\strut
\end{minipage} & \begin{minipage}[t]{0.23\columnwidth}\raggedright
in\strut
\end{minipage}\tabularnewline
\begin{minipage}[t]{0.23\columnwidth}\raggedleft
Volume\strut
\end{minipage} & \begin{minipage}[t]{0.23\columnwidth}\raggedright
37201\strut
\end{minipage} & \begin{minipage}[t]{0.23\columnwidth}\raggedright
gal\strut
\end{minipage}\tabularnewline
\begin{minipage}[t]{0.23\columnwidth}\raggedleft
Percent Filled\strut
\end{minipage} & \begin{minipage}[t]{0.23\columnwidth}\raggedright
53.3\strut
\end{minipage} & \begin{minipage}[t]{0.23\columnwidth}\raggedright
\%\strut
\end{minipage}\tabularnewline
\begin{minipage}[t]{0.23\columnwidth}\raggedleft
Mass\strut
\end{minipage} & \begin{minipage}[t]{0.23\columnwidth}\raggedright
74660\strut
\end{minipage} & \begin{minipage}[t]{0.23\columnwidth}\raggedright
lb\strut
\end{minipage}\tabularnewline
\begin{minipage}[t]{0.23\columnwidth}\raggedleft
Safety Valve Set\strut
\end{minipage} & \begin{minipage}[t]{0.23\columnwidth}\raggedright
155\strut
\end{minipage} & \begin{minipage}[t]{0.23\columnwidth}\raggedright
psig\strut
\end{minipage}\tabularnewline
\bottomrule
\end{longtable}

\begin{figure}[H]
\hypertarget{fig:fig_tank}{%
\begin{center}
\adjustimage{max size={0.9\linewidth}{0.9\paperheight},width=1.0\linewidth}{philadelphia_refinery_files/fig_tank.png}
\end{center}
\caption{Diagram depicting the dimensions of the TFSD tank. The tank was
positioned \(25ft\:(7.6m)\) above ground level and estimated to contain
\(20160gal\:(76.3m^3)\) (53.3\% filled) of butane (50\% by volume) and
butene (40\% by volume) at the time of the explosion.\cite{PES2019}}\label{fig:fig_tank}
}
\end{figure}

\begin{longtable}[]{@{}rr@{}}
\caption{Chemical Contents in the Treater-Feed-Surge-Drum at the Time of
the Explosion \cite{PES2019}
\label{tbl:tbl_chemicals_in_tank}}\tabularnewline
\toprule
\begin{minipage}[b]{0.19\columnwidth}\raggedleft
Chemical\strut
\end{minipage} & \begin{minipage}[b]{0.19\columnwidth}\raggedleft
Percent by Volume\strut
\end{minipage}\tabularnewline
\midrule
\endfirsthead
\toprule
\begin{minipage}[b]{0.19\columnwidth}\raggedleft
Chemical\strut
\end{minipage} & \begin{minipage}[b]{0.19\columnwidth}\raggedleft
Percent by Volume\strut
\end{minipage}\tabularnewline
\midrule
\endhead
\begin{minipage}[t]{0.19\columnwidth}\raggedleft
methane\strut
\end{minipage} & \begin{minipage}[t]{0.19\columnwidth}\raggedleft
0.01\strut
\end{minipage}\tabularnewline
\begin{minipage}[t]{0.19\columnwidth}\raggedleft
ethylene\strut
\end{minipage} & \begin{minipage}[t]{0.19\columnwidth}\raggedleft
0.00\strut
\end{minipage}\tabularnewline
\begin{minipage}[t]{0.19\columnwidth}\raggedleft
ethane\strut
\end{minipage} & \begin{minipage}[t]{0.19\columnwidth}\raggedleft
0.01\strut
\end{minipage}\tabularnewline
\begin{minipage}[t]{0.19\columnwidth}\raggedleft
propane\strut
\end{minipage} & \begin{minipage}[t]{0.19\columnwidth}\raggedleft
0.90\strut
\end{minipage}\tabularnewline
\begin{minipage}[t]{0.19\columnwidth}\raggedleft
propylene\strut
\end{minipage} & \begin{minipage}[t]{0.19\columnwidth}\raggedleft
0.10\strut
\end{minipage}\tabularnewline
\begin{minipage}[t]{0.19\columnwidth}\raggedleft
isobutane\strut
\end{minipage} & \begin{minipage}[t]{0.19\columnwidth}\raggedleft
37.28\strut
\end{minipage}\tabularnewline
\begin{minipage}[t]{0.19\columnwidth}\raggedleft
nbutane\strut
\end{minipage} & \begin{minipage}[t]{0.19\columnwidth}\raggedleft
12.81\strut
\end{minipage}\tabularnewline
\begin{minipage}[t]{0.19\columnwidth}\raggedleft
butens\strut
\end{minipage} & \begin{minipage}[t]{0.19\columnwidth}\raggedleft
40.41\strut
\end{minipage}\tabularnewline
\begin{minipage}[t]{0.19\columnwidth}\raggedleft
neopentane\strut
\end{minipage} & \begin{minipage}[t]{0.19\columnwidth}\raggedleft
0.00\strut
\end{minipage}\tabularnewline
\begin{minipage}[t]{0.19\columnwidth}\raggedleft
isopentane\strut
\end{minipage} & \begin{minipage}[t]{0.19\columnwidth}\raggedleft
3.94\strut
\end{minipage}\tabularnewline
\begin{minipage}[t]{0.19\columnwidth}\raggedleft
npentane\strut
\end{minipage} & \begin{minipage}[t]{0.19\columnwidth}\raggedleft
0.25\strut
\end{minipage}\tabularnewline
\begin{minipage}[t]{0.19\columnwidth}\raggedleft
butadiene\strut
\end{minipage} & \begin{minipage}[t]{0.19\columnwidth}\raggedleft
0.33\strut
\end{minipage}\tabularnewline
\begin{minipage}[t]{0.19\columnwidth}\raggedleft
benzene\strut
\end{minipage} & \begin{minipage}[t]{0.19\columnwidth}\raggedleft
0.00\strut
\end{minipage}\tabularnewline
\begin{minipage}[t]{0.19\columnwidth}\raggedleft
C5 olefins\strut
\end{minipage} & \begin{minipage}[t]{0.19\columnwidth}\raggedleft
3.54\strut
\end{minipage}\tabularnewline
\begin{minipage}[t]{0.19\columnwidth}\raggedleft
C6 sats\strut
\end{minipage} & \begin{minipage}[t]{0.19\columnwidth}\raggedleft
0.33\strut
\end{minipage}\tabularnewline
\begin{minipage}[t]{0.19\columnwidth}\raggedleft
C7+\strut
\end{minipage} & \begin{minipage}[t]{0.19\columnwidth}\raggedleft
0.04\strut
\end{minipage}\tabularnewline
\bottomrule
\end{longtable}

For this analysis, it will be assumed the entire mixture is butane. This
assumption avoids partial pressures and multiple energy calculations
with different vapor qualities greatly simplifying the analysis. The
assumption is reasonable because butane and butene are chemically
similar and make up more than 90\% of the fluid volume, see
\cref{tbl:tbl_butane_prop}. A BLEVE assumption is reasonable because,
the boiling temperatures for butene
\(\left(20.66\: ^{\circ} F\:(-6.3\: ^{\circ} C)\right)\) and butane
\(\left(31.1\: ^{\circ} F\:(-1.0\: ^{\circ} C)\right)\) are both well
above the tank temperature when exposed to fire.

\begin{longtable}[]{@{}rrr@{}}
\caption{Chemical Properties of Butene and Butane
\cite{PubChem-butane, PubChem-butene}
\label{tbl:tbl_butane_prop}}\tabularnewline
\toprule
\begin{minipage}[b]{0.27\columnwidth}\raggedleft
Property\strut
\end{minipage} & \begin{minipage}[b]{0.27\columnwidth}\raggedleft
Butene\strut
\end{minipage} & \begin{minipage}[b]{0.27\columnwidth}\raggedleft
Butane\strut
\end{minipage}\tabularnewline
\midrule
\endfirsthead
\toprule
\begin{minipage}[b]{0.27\columnwidth}\raggedleft
Property\strut
\end{minipage} & \begin{minipage}[b]{0.27\columnwidth}\raggedleft
Butene\strut
\end{minipage} & \begin{minipage}[b]{0.27\columnwidth}\raggedleft
Butane\strut
\end{minipage}\tabularnewline
\midrule
\endhead
\begin{minipage}[t]{0.27\columnwidth}\raggedleft
Molecular Formula\strut
\end{minipage} & \begin{minipage}[t]{0.27\columnwidth}\raggedleft
\(C_4H_8\)\strut
\end{minipage} & \begin{minipage}[t]{0.27\columnwidth}\raggedleft
\(C_4H_{10}\)\strut
\end{minipage}\tabularnewline
\begin{minipage}[t]{0.27\columnwidth}\raggedleft
Molecular Weight\strut
\end{minipage} & \begin{minipage}[t]{0.27\columnwidth}\raggedleft
\(56.108\frac{g}{mol}\)\strut
\end{minipage} & \begin{minipage}[t]{0.27\columnwidth}\raggedleft
\(58.12g\frac{g}{mol}\)\strut
\end{minipage}\tabularnewline
\begin{minipage}[t]{0.27\columnwidth}\raggedleft
Boiling Point at 760mm Hg\strut
\end{minipage} & \begin{minipage}[t]{0.27\columnwidth}\raggedleft
\(20.6^{\circ}F\)\strut
\end{minipage} & \begin{minipage}[t]{0.27\columnwidth}\raggedleft
\(31.1^{\circ}F\)\strut
\end{minipage}\tabularnewline
\begin{minipage}[t]{0.27\columnwidth}\raggedleft
Flash Point\strut
\end{minipage} & \begin{minipage}[t]{0.27\columnwidth}\raggedleft
\(-110.0^{\circ}F\)\strut
\end{minipage} & \begin{minipage}[t]{0.27\columnwidth}\raggedleft
\(-76.0^{\circ}F\)\strut
\end{minipage}\tabularnewline
\begin{minipage}[t]{0.27\columnwidth}\raggedleft
Density at 25\(^{\circ} C\), 1 atm\strut
\end{minipage} & \begin{minipage}[t]{0.27\columnwidth}\raggedleft
\(0.588\frac{g}{cc}\)\strut
\end{minipage} & \begin{minipage}[t]{0.27\columnwidth}\raggedleft
\(0.573\frac{g}{cc}\)\strut
\end{minipage}\tabularnewline
\begin{minipage}[t]{0.27\columnwidth}\raggedleft
\(\frac{C_P}{C_V}\) Ratio \(\gamma\)\strut
\end{minipage} & \begin{minipage}[t]{0.27\columnwidth}\raggedleft
1.13\strut
\end{minipage} & \begin{minipage}[t]{0.27\columnwidth}\raggedleft
1.12\strut
\end{minipage}\tabularnewline
\bottomrule
\end{longtable}

\hypertarget{pressure-relief-valve}{%
\subsection{Pressure Relief Valve}\label{pressure-relief-valve}}

The TFSD was fitted with a Consolidated (1906-30LC-1-CC-MS-31-RF-1)
\(3in\:x\:4in\:(7.63cm\:x\:10.16cm)\) pressure relief valve (PRV). The
relief pressure was set to \(155 psig\:(1068.7kPa)\) and the relief
temperature was set to \(183.5^{\circ} F\:(84.3^{\circ}C)\). The PRV was
positioned on the top of the TFSD, see \cref{fig:fig_tank}. It is
assumed the PRV functioned as designed.

\hypertarget{recovered-drum-debris}{%
\subsection{Recovered Drum Debris}\label{recovered-drum-debris}}

Three major pieces of the TFSD were identified after the explosion, see
\cref{fig:blast_debris}. The largest piece (large-end-cap) was recovered
\(2100ft\:(640m)\) from the blast seat. It was approximately
\(22ft\:(6.7m)\) in length, or a little more than half the original tank
length, see \cref{fig:large_end_cap}. The other end of the TFSD
(small-end-cap) was recovered \(1761ft\:(536m)\) from the blast seat. It
was approximately \(5ft\:(1.5m)\) in length, see
\cref{fig:small_end_cap}. The piece thrown the shortest distance
(center-section) was \(819ft\:(249m)\) from the blast seat. The
center-section piece was heavily damaged and a photographic analysis of
the length was not possible. However, based on the original length of
the tank after removing the large and small-end-cap lengths, the
center-section length is approximately \(12ft\:(3.6m)\), see
\cref{fig:fillet}.

\begin{figure}[H]
\hypertarget{fig:blast_debris}{%
\begin{center}
\adjustimage{max size={0.9\linewidth}{0.9\paperheight},width=1.0\linewidth}{philadelphia_refinery_files/drum_debris_and_blast_seat_distances.jpg}
\end{center}
\caption{Map of Girard Point showing the blast seat and locations of the three
main pieces of the TSFD (V-1) recovered.\cite{Malone2019a}}\label{fig:blast_debris}
}
\end{figure}

\begin{figure}[H]
\hypertarget{fig:large_end_cap}{%
\begin{center}
\adjustimage{max size={0.9\linewidth}{0.9\paperheight},width=1.0\linewidth}{philadelphia_refinery_files/tank_length_estimate.png}
\end{center}
\caption{Photograph of the large-end-cap, estimated to be \(22ft\:(6.7m)\) in
length based on photographic analysis. In this photograph the tank has
been moved from it's original landing location.\cite{Malone2019a}}\label{fig:large_end_cap}
}
\end{figure}

\begin{figure}[H]
\hypertarget{fig:small_end_cap}{%
\begin{center}
\adjustimage{max size={0.9\linewidth}{0.9\paperheight},width=1.0\linewidth}{philadelphia_refinery_files/end_cap_03.jpg}
\end{center}
\caption{Photograph of the small-end-cap, estimated to be \(5ft\:(1.5m)\) in
length based on photographic analysis.\cite{Malone2019a}}\label{fig:small_end_cap}
}
\end{figure}

\begin{figure}[H]
\hypertarget{fig:fillet}{%
\begin{center}
\adjustimage{max size={0.9\linewidth}{0.9\paperheight},width=1.0\linewidth}{philadelphia_refinery_files/1_fish_fillet2.jpg}
\end{center}
\caption{Photograph of the center-section piece, calculated to be
\(12ft\:(3.6m)\) in length based on the total length of the drum less
the lengths of the large and small-end-caps.\cite{Malone2019a}}\label{fig:fillet}
}
\end{figure}

The estimated mass of each piece of the TFSD is summarized in
\cref{tbl:tbl_tank_wt}. The mass calculations are based on the refinery
recorded mass of the vessel, \(74660lb\:(33865kg)\), multiplied by the
percent mass calculated using the following equations,

Ellipsoidal Head:
\begin{equation}V_{eh}=\frac{\pi D^2 (D/4)}{6}\end{equation} Cylinder:
\begin{equation}V_c = \frac{\pi D^2 l}{4}\end{equation}

where the drum wall thickness is \(0.83in\:(2.1cm)\), the tank diameter
is \(12ft\:(3.7m)\), and the material density for A516 steel is
\(7.8g/cc\)\cite{Malone2019}. The difference in calculated and refinery
recorded mass is likely due to other structural components attached to
the tank not accounted for in the calculated mass.

\begin{longtable}[]{@{}lrrrr@{}}
\caption{Estimated TFSD Debris Mass \cite{PES2019}
\label{tbl:tbl_tank_wt}}\tabularnewline
\toprule
\begin{minipage}[b]{0.17\columnwidth}\raggedright
TFSD ID\strut
\end{minipage} & \begin{minipage}[b]{0.09\columnwidth}\raggedleft
Cylinder (ft)\strut
\end{minipage} & \begin{minipage}[b]{0.09\columnwidth}\raggedleft
Calculated Debris Mass (lb)\strut
\end{minipage} & \begin{minipage}[b]{0.13\columnwidth}\raggedleft
Percent Mass Calculated\strut
\end{minipage} & \begin{minipage}[b]{0.13\columnwidth}\raggedleft
Refinery Recorded Mass (lb)\strut
\end{minipage}\tabularnewline
\midrule
\endfirsthead
\toprule
\begin{minipage}[b]{0.17\columnwidth}\raggedright
TFSD ID\strut
\end{minipage} & \begin{minipage}[b]{0.09\columnwidth}\raggedleft
Cylinder (ft)\strut
\end{minipage} & \begin{minipage}[b]{0.09\columnwidth}\raggedleft
Calculated Debris Mass (lb)\strut
\end{minipage} & \begin{minipage}[b]{0.13\columnwidth}\raggedleft
Percent Mass Calculated\strut
\end{minipage} & \begin{minipage}[b]{0.13\columnwidth}\raggedleft
Refinery Recorded Mass (lb)\strut
\end{minipage}\tabularnewline
\midrule
\endhead
\begin{minipage}[t]{0.17\columnwidth}\raggedright
Large-End-Cap\strut
\end{minipage} & \begin{minipage}[t]{0.09\columnwidth}\raggedleft
22.0\strut
\end{minipage} & \begin{minipage}[t]{0.09\columnwidth}\raggedleft
31541.8\strut
\end{minipage} & \begin{minipage}[t]{0.13\columnwidth}\raggedleft
0.56\strut
\end{minipage} & \begin{minipage}[t]{0.13\columnwidth}\raggedleft
41809.6\strut
\end{minipage}\tabularnewline
\begin{minipage}[t]{0.17\columnwidth}\raggedright
Small-End-Cap\strut
\end{minipage} & \begin{minipage}[t]{0.09\columnwidth}\raggedleft
5.0\strut
\end{minipage} & \begin{minipage}[t]{0.09\columnwidth}\raggedleft
10078.5\strut
\end{minipage} & \begin{minipage}[t]{0.13\columnwidth}\raggedleft
0.18\strut
\end{minipage} & \begin{minipage}[t]{0.13\columnwidth}\raggedleft
13438.8\strut
\end{minipage}\tabularnewline
\begin{minipage}[t]{0.17\columnwidth}\raggedright
Center-Section\strut
\end{minipage} & \begin{minipage}[t]{0.09\columnwidth}\raggedleft
12.0\strut
\end{minipage} & \begin{minipage}[t]{0.09\columnwidth}\raggedleft
15150.5\strut
\end{minipage} & \begin{minipage}[t]{0.13\columnwidth}\raggedleft
0.26\strut
\end{minipage} & \begin{minipage}[t]{0.13\columnwidth}\raggedleft
19411.6\strut
\end{minipage}\tabularnewline
\begin{minipage}[t]{0.17\columnwidth}\raggedright
Total\strut
\end{minipage} & \begin{minipage}[t]{0.09\columnwidth}\raggedleft
39.0\strut
\end{minipage} & \begin{minipage}[t]{0.09\columnwidth}\raggedleft
56770.8\strut
\end{minipage} & \begin{minipage}[t]{0.13\columnwidth}\raggedleft
1.00\strut
\end{minipage} & \begin{minipage}[t]{0.13\columnwidth}\raggedleft
74660.0\strut
\end{minipage}\tabularnewline
\bottomrule
\end{longtable}

\hypertarget{boiling-liquid-expanding-vapor-explosion-bleve}{%
\subsection{Boiling-Liquid Expanding-Vapor Explosion
(BLEVE)}\label{boiling-liquid-expanding-vapor-explosion-bleve}}

A BLEVE results from the sudden failure of a tank containing a
compressed vapor (head space) and a super-heated liquid (a liquid heated
above it's boiling point but without boiling). The magnitude of the
blast depends on how super-heated the liquid was at failure. There is a
direct relationship between the super-heat temperature and the quantity
of liquid that flash-boils. Higher volumes of fluid that flash-boil
release more energy. Once containment failure occurs, the energy is
distributed into four forms:

\begin{enumerate}
\def\labelenumi{\arabic{enumi}.}
\tightlist
\item
  Overpressure wave
\item
  Kinetic energy of fragments
\item
  Deformation and failure of the containment material
\item
  Heat transferred to environment
\end{enumerate}

The distribution of the energy into the these four forms depends on the
specifics of the explosion. Planas-Cuchi et al.~found that a
\emph{brittle}\footnote{Brittle failure refers to the breakage of a material due to a sudden fracture. When a brittle failure occurs, the material breaks suddenly instead of deforming or straining under load. The fracturing or breaking can occur with only a small amount of load, impact force or shock. Brittle materials absorb less energy before breaking or fracturing, despite the materials having a high strength.}
failure releases 80\% of the energy into the blastwave, while a
\emph{ductile}\footnote{A ductile failure is a type of failure seen in malleable materials characterized by extensive plastic deformation or necking. This usually occurs prior to the actual failure of the material. The term ductile rupture refers to the failure of highly ductile materials. In such cases, materials pull apart instead of cracking.}
failure releases 40\% of the energy into the blastwave. The remaining
energy becomes kinetic energy of the fragments. The heat transfer to the
environment is relatively small \cite{Planas2004}. In practice, most
pressure vessels are designed with materials that are ductile rather
than brittle to avoid sudden and catastrophic brittle (fragile) failures
\cite{Benac2016}.

\hypertarget{overpressure-and-impulse-from-the-bleve}{%
\section{Overpressure and Impulse from the
BLEVE}\label{overpressure-and-impulse-from-the-bleve}}

There is a seven step method for calculating the overpressure and
impulse from a BLEVE. The method is given in
\cref{fig:fig_bleve_process}.

\begin{figure}[H]
\hypertarget{fig:fig_bleve_process}{%
\begin{center}
\adjustimage{max size={0.9\linewidth}{0.9\paperheight},width=0.35\linewidth}{philadelphia_refinery_files/fig_bleve_airblast_calc.png}
\end{center}
\caption{Calculation of energy of flashing liquids and pressure vessel bursts
filled with vapor or nonideal gas. \cite{Safety2010}}\label{fig:fig_bleve_process}
}
\end{figure}

\hypertarget{data-collection}{%
\subsection{Data Collection}\label{data-collection}}

For this analysis the following data will be used:

\begin{enumerate}
\def\labelenumi{\arabic{enumi}.}
\tightlist
\item
  Ambient air pressure, \(14.7psi\:(101.3kPa)\)
\item
  Vessel volume, \(37200.63gal\:(140.82m^3)\)
\item
  Ratio of specific heats of butane (1.12)
\item
  Distance from the center of the tank to the receptor,
  \(32.81ft\:(10.00m)\)
\item
  Shape of the vessel is cylindrical, \(L/D=3.29\)
\item
  Speed of sound of air, \(1115.49ft/s\:(340.00m/s)\)
\end{enumerate}

\hypertarget{internal-energy}{%
\subsection{Internal Energy}\label{internal-energy}}

\hypertarget{pressure-at-state-1-pre-failure-state}{%
\subsubsection{Pressure at State 1 (Pre-failure
State)}\label{pressure-at-state-1-pre-failure-state}}

The TFSD is assumed to fail at \(1.21\) times the opening pressure of
the pressure relief valve (PRV)\cite{Engineers2013}. This pressure is
based on the American Petroleum Institutes Standard 521 requiring
pressure relief valves on pressure vessels to achieve rated flow at 1.21
times the maximum allowable working pressure. The PRV was set to
\(155.00psi\:(1068.69kPa)\) therefore, the absolute pressure at state 1
(failure state) is given by,

\begin{equation}p_1 = 1.21\left(p_{PRV}+p_{atm}\right)\end{equation}
\begin{equation}p_1 = 1.21\left(155.00+14.70\right)\end{equation}
\begin{equation}p_1 = 205.33 psia\:(1.42\:MPa)\end{equation}

\hypertarget{pressure-at-state-2-final-expanded-state}{%
\subsubsection{Pressure at State 2 (Final Expanded
State)}\label{pressure-at-state-2-final-expanded-state}}

The pressure at state 2 (final expanded state) is standard atmospheric
pressure or \(14.7psi\:(101.3kPa)\).

\hypertarget{internal-energy-1}{%
\subsubsection{Internal Energy}\label{internal-energy-1}}

The internal energy \(u\) can be used to estimate the energy released in
an explosion. With the gases in the saturated state and knowing the
pressures at state 1 (explosion) and state 2 (atmospheric), lookup
tables can be used to determine the specific volume \(v\), and the
enthalpy \(h\). Combining these two properties with the pressure, the
internal energy \(u\) can be calculated using,

\begin{equation}h = u + pv\end{equation}

where \(h\) is the enthalpy, \(p\) is the pressure, and \(v\) is the
specific volume. Therefore, solving for the internal energy \(u\) we
have,

\begin{equation}u = h - pv\end{equation}

\hypertarget{internal-energy-at-state-1}{%
\paragraph{Internal Energy at State
1}\label{internal-energy-at-state-1}}

The internal energy at state 1 for saturated liquid butane is,

\begin{equation}u_{1f} = h_{1f} - (p_{1})(v_{1f})\end{equation}
\begin{equation}u_{1f} = 451.46\:kJ/kg - (1415.72\:kPa)(0.002110\:m^3/kg)\end{equation}
\begin{equation}u_{1f} = 448.47\:kJ/kg\end{equation}

and for the saturated vapor butane at state 1 we have,

\begin{equation}u_{1g} = h_{1g} - (p_{1})(v_{1g})\end{equation}
\begin{equation}u_{1g} = 716.93\:kJ/kg - (1415.72\:kPa)(0.027658\:m^3/kg)\end{equation}
\begin{equation}u_{1g} = 677.77\:kJ/kg\end{equation}

and similarly for saturated liquid and vapor butane at state 2,

\begin{equation}u_{2f} = 198.70\:kJ/kg\end{equation}
\begin{equation}u_{2g} = 547.18\:kJ/kg\end{equation}

All the gas properties are summarized in \cref{tbl:thermo}

\begin{table}[H]
\caption{Propane Thermodynamic Data for Inital (1) and Final (2) States}\label{tbl:thermo}
\centering
\begin{adjustbox}{max width=\textwidth}
\begin{tabular}{lrrrrrrrrr}
\toprule
{} &  $P \left(kPa\right)$ &  $u_f \left(\frac{kJ}{kg\:K}\right)$ &  $u_g \left(\frac{kJ}{kg\:K}\right)$ &  $v_f \left(\frac{m^3}{kg}\right)$ &  $v_g \left(\frac{m^3}{kg}\right)$ &  $s_f \left(\frac{kJ}{kg\:K}\right)$ &  $s_g \left(\frac{kJ}{kg\:K}\right)$ &  $h_f \left(\frac{kJ}{kg}\right)$ &  $h_g \left(\frac{kJ}{kg}\right)$ \\
State    &                       &                                      &                                      &                                    &                                    &                                      &                                      &                                   &                                   \\
\midrule
butane-1 &              1.42E+03 &                             4.48E+02 &                             6.78E+02 &                           2.11E-03 &                           2.77E-02 &                             1.78E+00 &                             2.49E+00 &                          4.51E+02 &                          7.17E+02 \\
butane-2 &              1.01E+02 &                             1.99E+02 &                             5.47E+02 &                           1.66E-03 &                           3.69E-01 &                             9.96E-01 &                             2.41E+00 &                          1.99E+02 &                          5.85E+02 \\
\bottomrule
\end{tabular}

\end{adjustbox}
\end{table}

\hypertarget{internal-energy-at-state-2}{%
\paragraph{Internal Energy at State
2}\label{internal-energy-at-state-2}}

When the drum containment fails and the butane at state 1 expands to
state 2 (atmospheric pressure), it is no longer at a saturated stated.
Some of the liquid vaporizes and some of the vapor condenses. The
quality of the vapor and liquid can be used to calculate the unsaturated
internal energy at state 2. The vapor quality \((\chi)\), can be
calculated from,

\begin{equation}\chi = \frac{s_1 - s_2}{s_{2g}-s_{2f}}\end{equation}

where \(s_1\) and \(s_2\) are the entropies at the initial state (1) and
final state (2) and \(s_{2g}\) and \(s_{2f}\) are the entropies of the
final state (2) of the saturated vapor and fluid respectively. If
calculating the quality of the saturated \emph{liquid}, then
\(s_1=s_{1f}\) and \(s_2=s_{2f}\) where \(s_{1f}\) and \(s_{2f}\) are
the entropies at the initial (1) and final (2) states of the saturated
fluid. Similarly, if calculating the quality of the saturated
\emph{vapor} \(s_1=s_{1g}\) and \(s_2 = s_{2g}\) where \(s_{1g}\) and
\(s_{2g}\) are the entropies at the initial (1) and final (2) states of
the saturated vapor. Therefore, for the saturated liquid butane,

\begin{equation}\chi_f = \frac{s_{1f} - s_{2f}}{s_{2g}-s_{2f}}\end{equation}

\begin{equation}\chi_f = \frac{1.775 - 0.996}{2.411-0.996}\end{equation}

\begin{equation}\chi_f = 0.552\end{equation}

and for the saturated vapor,

\begin{equation}\chi_g = \frac{s_{1g} - s_{2g}}{s_{2g}-s_{2f}}\end{equation}

\begin{equation}\chi_g= \frac{2.495 - 2.410}{2.410-0.996}\end{equation}

\begin{equation}\chi_g = 0.060\end{equation}

The unsaturated internal energy at state 2 can be calculated using,

\begin{equation}u_{2f} = (1-\chi_f)u_{2f} + \chi_f u_{2g}\end{equation}
\begin{equation}u_{2g} = (1-\chi_g)u_{2g} + \chi_g u_{2f}\end{equation}

\begin{equation}u_{2f} = (1-0.552)198.70 + (0.552)(547.18)\end{equation}
\begin{equation}u_{2g} = (1-0.060)547.18 + (0.060)(198.70)\end{equation}

\begin{equation}u_{2f} = 390.89\:kJ/kg\end{equation}
\begin{equation}u_{2g} = 526.40\:kJ/kg\end{equation}

\hypertarget{specific-work}{%
\subsection{Specific Work}\label{specific-work}}

The specific work done by a BLEVE is the difference in internal energies
at state 1 and state 2, \begin{equation}e_{ex}=u_1-u_2\end{equation} For
the specific work of the liquid butane,
\begin{equation}e_{ex-f} = u_{1f} - u_{2f}\end{equation}
\begin{equation}e_{ex-f} = 448.47\:kJ/kg - 390.89\:kJ/kg\end{equation}
\begin{equation}e_{ex-f} = 57.58\:kJ/kg\end{equation} For the specific
work of the vapor butane we have,
\begin{equation}e_{ex-g} = u_{1-g} - u_{2-g}\end{equation}
\begin{equation}e_{ex-g} = 677.77\:kJ/kg - 526.40\:kJ/kg\end{equation}
\begin{equation}e_{ex-g} = 151.38\:kJ/kg\end{equation}

\hypertarget{explosion-energy}{%
\subsection{Explosion Energy}\label{explosion-energy}}

The explosion energy is the specific work multiplied by the mass of
fluid (liquid or vapor) initially in the TFSD.

\begin{equation}E_{ex}=e_{ex}m\end{equation}

Where \(e_{ex}\) is the specific work of the fluid (vapor or liquid) and
\(m\) is the mass of the fluid (vapor or liquid).

However, there are two additional factors that must be considered when
estimating the explosion energy. The first, is the ground reflection
factor, \(gnd\). If the TFSD were on the ground the reflection factor
would be \(gnd=2\) however, the TFSD was located \(25ft\:(7.62m)\) above
ground level. For this case, a reflection factor of \(gnd=1.25\) will be
assumed. The second factor is estimating the amount of energy lost in
fragmenting the tank, \(frag\). This reduction in energy can range from
\(frag = 20\%\)-\(50\%\) \cite{Safety2010}. For this calculation, a
fragmentation factor of \(frag=40\%\) will be assumed, therefore
\(60\%\) is available for overpressure and impulse. With these
additional factors our explosion energy is,

\begin{equation}E_{ex}=(gnd)(frag)e_{ex}m\end{equation}
\begin{equation}E_{ex}=(1.25)(0.6)e_{ex}m\end{equation}
\begin{equation}E_{ex}=0.75e_{ex}m\end{equation}

Because the fluid is present both as a saturated vapor and liquid, the
explosion energy for each must be calculated separately and then summed
together. For the saturated liquid butane,

\begin{equation}E_{ex-f}=0.75(57.58kJ/kg)(35570.96kg)\end{equation}
\begin{equation}E_{ex-f}=1536.08MJ\end{equation}

and similarly for the saturated vapor,

\begin{equation}E_{ex-g}=269.95MJ\end{equation}

Therefore the total energy for explosion is,

\begin{equation}E_{tot} = 1806.03MJ\end{equation}

\hypertarget{non-dimensional-range-to-the-target}{%
\subsection{Non-dimensional Range to the
Target}\label{non-dimensional-range-to-the-target}}

For this analysis, a range of \(R= 32.18ft\:(10m)\) was chosen. However,
to use the Baker-Tang overpressure and impulse curves, we must calculate
the non-dimensional range to the target from\cite{Safety2010},

\begin{equation}\overline{R} = R\left(\frac{p_0}{E_{tot}}\right)^{1/3}\end{equation}

where \(R\) is the range where you would like the pressure and impulse
calculated, \(p_0\) is atmospheric pressure, and \(E_{tot}\) is the
total explosion energy calculated previously.

\begin{equation}\overline{R} = 10m\left(\frac{101325Pa}{1806.03x10^6}\right)^{1/3} = 0.38\end{equation}

With the non-dimensional range calculated, the Baker-Tang overpressure
and impulse curves can be used to calculate the non-dimensional
overpressure and impulse.

\hypertarget{non-dimensional-side-on-pressure-and-impulse}{%
\subsection{Non-dimensional Side-on Pressure and
Impulse}\label{non-dimensional-side-on-pressure-and-impulse}}

The non-dimensional side-on pressure can be calculated from
\cref{fig:overpressure} and gives a \(\bar{P} = 0.950\) for an
\(\bar{R}=0.38\) and a \(p/p_0 = 13.97 \approx 10\). The non-dimensional
side-on impulse can be calculated from \cref{fig:impulse} and gives a
\(\bar{I} = 0.107\) for an \(\bar{R}=0.38\) and a
\(p/p_0 = 13.97 \approx 10\).

\hypertarget{side-on-pressure-and-impulse}{%
\subsection{Side-on Pressure and
Impulse}\label{side-on-pressure-and-impulse}}

The dimensional or \emph{real} side-on pressure and impulse can be
calculated from the following:

\begin{equation}P_s = \bar{P}p_0 = (0.950)(101.325\:kPa) = 96.26\:kPa \:(13.96psi)\end{equation}

\begin{equation}i_s = \frac{\left(\bar I p_0^{2/3}E_{tot}^{1/3}\right)}{a_0}=\frac{(0.107)(101325\:Pa)^{2/3}(1806.03x10^6\:J)^{1/3}}{340\:m/s}= 831.08Pa {\text -} s\:(0.12psi{\text -}s)\end{equation}

This calculation only accounts for the blast from the expansion of the
vessel contents. The blast may be followed by a vapor cloud explosion.
In this case, when the expanding contents of the vessel are immediately
exposed to fire, there is very little additional overpressure generated.
This is because the expanding contents begin burning as soon as the fuel
air ratio will support combustion.

\hypertarget{direct-effects-of-blast-overpressure-on-the-human-body}{%
\subsection{Direct Effects of Blast Overpressure on the Human
Body}\label{direct-effects-of-blast-overpressure-on-the-human-body}}

The direct effects of blast overpressure are due to the positive and
negative phases of the shock wave as it passes through the human body.
Damage to tissue is due to the shock wave passing through tissue with
different densities. These density changes place tissue in compression
and tension resulting in hemorrhages and air embolisms. Typical blast
over pressure injuries for a human in the open are summarized in
\cref{tbl:tbl_nfpa}. A person positioned \(32.8ft\:(10m)\) from the TFSD
would experience an absolute pressure of \(28.66psia\:(197603.74Pa)\)
and have a 99\% chance of fatality. The ``threshold for fatality'' would
occur at a distance of approximately \(869.4ft\:(265m)\). \newpage

\begin{longtable}[]{@{}ll@{}}
\caption{Overpressure vs.~Human Injury Probability\cite{nfpa_921_2014}
\label{tbl:tbl_nfpa}}\tabularnewline
\toprule
\begin{minipage}[b]{0.24\columnwidth}\raggedright
Overpressure (psig)\strut
\end{minipage} & \begin{minipage}[b]{0.38\columnwidth}\raggedright
Injury\strut
\end{minipage}\tabularnewline
\midrule
\endfirsthead
\toprule
\begin{minipage}[b]{0.24\columnwidth}\raggedright
Overpressure (psig)\strut
\end{minipage} & \begin{minipage}[b]{0.38\columnwidth}\raggedright
Injury\strut
\end{minipage}\tabularnewline
\midrule
\endhead
\begin{minipage}[t]{0.24\columnwidth}\raggedright
14.5\strut
\end{minipage} & \begin{minipage}[t]{0.38\columnwidth}\raggedright
Threshold for fatality\strut
\end{minipage}\tabularnewline
\begin{minipage}[t]{0.24\columnwidth}\raggedright
16.0\strut
\end{minipage} & \begin{minipage}[t]{0.38\columnwidth}\raggedright
50\% ear drum rupture\strut
\end{minipage}\tabularnewline
\begin{minipage}[t]{0.24\columnwidth}\raggedright
17.5\strut
\end{minipage} & \begin{minipage}[t]{0.38\columnwidth}\raggedright
10\% probability for fatality\strut
\end{minipage}\tabularnewline
\begin{minipage}[t]{0.24\columnwidth}\raggedright
20.5\strut
\end{minipage} & \begin{minipage}[t]{0.38\columnwidth}\raggedright
50\% probability for fatality\strut
\end{minipage}\tabularnewline
\begin{minipage}[t]{0.24\columnwidth}\raggedright
25.5\strut
\end{minipage} & \begin{minipage}[t]{0.38\columnwidth}\raggedright
90\% probability for fatality\strut
\end{minipage}\tabularnewline
\begin{minipage}[t]{0.24\columnwidth}\raggedright
29.0\strut
\end{minipage} & \begin{minipage}[t]{0.38\columnwidth}\raggedright
99\% probability for fatality\strut
\end{minipage}\tabularnewline
\bottomrule
\end{longtable}

\begin{figure}[H]\begin{center}\adjustimage{max size={0.9\linewidth}{0.9\paperheight},height=0.33\paperheight}{philadelphia_refinery_files/output_37_0.png}\end{center}\caption{Non-dimensional overpressure curves for various vessel internal
pressures. The arrow designates the non-dimensional pressure and range
point for a butane BLEVE in the TFSD.}\label{fig:overpressure}\end{figure}

\begin{figure}[H]\begin{center}\adjustimage{max size={0.9\linewidth}{0.9\paperheight},height=0.33\paperheight}{philadelphia_refinery_files/output_41_0.png}\end{center}\caption{Non-dimenstional impulse curves for various vessel internal pressures.
The arrow designates the non-dimensional impulse and range point for a
butane BLEVE in the TFSD.}\label{fig:impulse}\end{figure}

\hypertarget{debris-from-the-bleve}{%
\section{Debris from the BLEVE}\label{debris-from-the-bleve}}

The same explosion energy results developed in the overpressure and
impulse calculations can be used for the fragmentation analysis.
However, the overpressure energy calculations applied a \(grd=1.25\)
ground reflection factor. This factor is not applicable when determining
the energy available for fragment throw calculations. However, the
\(frag=0.4\) fragmentation factor should be applied. Therefore, the
available energy for fragmentation is,

\begin{equation}E_{ex-f}=0.4(57.58kJ/kg)(35570.96kg)\end{equation}
\begin{equation}E_{ex-f}=819.24MJ\end{equation}

and similarly for the saturated vapor,

\begin{equation}E_{ex-g}=143.97MJ\end{equation}

Therefore the total energy for fragmentation is,

\begin{equation}E_{tot} = 963.22MJ\end{equation}

\hypertarget{initial-debris-speed}{%
\subsection{Initial Debris Speed}\label{initial-debris-speed}}

A conservative method for determining the initial speed of fragments is
the empirical method proposed by Moore\cite{moore1967}. The initial
velocity is given by:

\begin{equation}v_i = 1.0092\left(\frac{E_{tot}G}{M_C}\right)^{0.5}\end{equation}

where for a cylindrical vessel,

\begin{equation}G = \frac{1}{1+M_G/2M_C}\end{equation}

and \(M_G\) is the total gas mass, \(E_{tot}\) is the energy, and
\(M_C\) is the mass of the vessel. For our case we have,

\begin{equation}v_i = 1.092\left(\frac{(963.22E6J)(0.60)}{35570.96kg}\right)^{0.5}\end{equation}

\begin{equation}v_i = 318.85 mph\:(142.54 m/s)\end{equation}

\hypertarget{debris-throw-ranges}{%
\subsection{Debris Throw Ranges}\label{debris-throw-ranges}}

With the initial debris speed calculated and the distances to the three
TFSD pieces known, we can calculate the required launch angle. The
initial trajectory angle \((a_i)\) can be calculated (neglecting lift
and drag forces) using,

\begin{equation}a_i = asin\left(\frac{Rg}{2v_i^2}\right)\end{equation}

where \(R\) is the horizontal range and \(g\) is the acceleration of
gravity.

For the initial velocity of \(318.85mph\:(142.54m/s)\) the large-end-cap
at a range of \(2099.7ft\:(640m)\) would need a launch angle of
\(9.0^\circ\), the small-end-cap at a range of \(1715.8ft\:(523m)\)
would need a launch angle of \(7.3^\circ\), and the center-section at a
range of \(816.9ft\:(249m)\) would need a launch of angle of
\(3.5^\circ\), see \cref{fig:fig_launch_angle}. Therefore, the initial
velocity predicted supports the throw ranges observed. The maximum throw
range possible (assuming no aerodynamic lift) occurs at a launch angle
of \(45^{\circ}\). For this case, the maximum throw range would be
\(6794.9ft\:(2071.1m)\).

\begin{figure}[H]\begin{center}\adjustimage{max size={0.9\linewidth}{0.9\paperheight},height=0.33\paperheight}{philadelphia_refinery_files/output_54_0.png}\end{center}\caption{Range of debris pieces and calculated launch angle for an initial speed
of 318.85mph (142.54m/s).}\label{fig:fig_launch_angle}\end{figure}

\hypertarget{thermal-radiation-from-the-bleve}{%
\section{Thermal Radiation from the
BLEVE}\label{thermal-radiation-from-the-bleve}}

\hypertarget{fireball-size-and-duration}{%
\subsection{Fireball Size and
Duration}\label{fireball-size-and-duration}}

An estimate for the fireball diameter \(D_c\) and duration \(t_c\)
generated from a BLEVE of \(78420.54lb\:(35570.96kg)\) of liquid butane
can be calculated using the equation,

\begin{equation}D_c = 5.8m_f^{1/3}\end{equation}

where \(m_f\) is the mass of fluid. For this case,

\begin{equation}D_c = 5.8(35570.96)^{1/3}\end{equation}
\begin{equation}D_c = 625.81ft\:(190.75m)\end{equation}

see \cref{fig:fig_fireball_dia} for a scaled drawing showing the
fireball size. The fireball duration \(t_c\) can be estimated from,

\begin{equation}t_c = 0.45m_f^{1/3}\end{equation}
\begin{equation}t_c = 0.45(35570.96)^{1/3}\end{equation}
\begin{equation}t_c = 14.80s\end{equation}

Therefore, the fireball from the TFSD would have expanded to
\(625.81ft\:(190.75m)\) and then dissipated in \(14.8s\).

\begin{figure}[H]
\hypertarget{fig:fig_fireball_dia}{%
\begin{center}
\adjustimage{max size={0.9\linewidth}{0.9\paperheight},width=1.0\linewidth}{philadelphia_refinery_files/fig_fire_ball.png}
\end{center}
\caption{Diagram depicting the estimated fireball diameter,
\(625.8ft\:(190.7m)\), from a butane BLEVE originating from the TFSD
tank. The TFSD tank and a 6ft human are shown for scale.}\label{fig:fig_fireball_dia}
}
\end{figure}

\hypertarget{thermal-radiation}{%
\subsection{Thermal Radiation}\label{thermal-radiation}}

An estimate for the thermal (infrared) radiation from the BLEVE can be
estimated for a standing (vertical) observer some distance from the
fireball. The thermal radiation can then be converted to thermal dose
units \((TDU)\) to determine the range where the thresholds for pain and
1st, 2nd, and 3rd degree burns would be observed. The critical TDUs are
summarized in \cref{tbl:tbl_TDU}.

\begin{longtable}[]{@{}lr@{}}
\caption{Burn Injury vs.~Thermal Dose Relationship \cite{OSullivan2004}
\label{tbl:tbl_TDU}}\tabularnewline
\toprule
\begin{minipage}[b]{0.33\columnwidth}\raggedright
Harm Caused\strut
\end{minipage} & \begin{minipage}[b]{0.24\columnwidth}\raggedleft
Mean TDU \(((kW/m^2)^{4/3}s)\)\strut
\end{minipage}\tabularnewline
\midrule
\endfirsthead
\toprule
\begin{minipage}[b]{0.33\columnwidth}\raggedright
Harm Caused\strut
\end{minipage} & \begin{minipage}[b]{0.24\columnwidth}\raggedleft
Mean TDU \(((kW/m^2)^{4/3}s)\)\strut
\end{minipage}\tabularnewline
\midrule
\endhead
\begin{minipage}[t]{0.33\columnwidth}\raggedright
Pain\strut
\end{minipage} & \begin{minipage}[t]{0.24\columnwidth}\raggedleft
92\strut
\end{minipage}\tabularnewline
\begin{minipage}[t]{0.33\columnwidth}\raggedright
Threshold 1st Degree Burns\strut
\end{minipage} & \begin{minipage}[t]{0.24\columnwidth}\raggedleft
105\strut
\end{minipage}\tabularnewline
\begin{minipage}[t]{0.33\columnwidth}\raggedright
Threshold 2nd Degree Burns\strut
\end{minipage} & \begin{minipage}[t]{0.24\columnwidth}\raggedleft
290\strut
\end{minipage}\tabularnewline
\begin{minipage}[t]{0.33\columnwidth}\raggedright
Threshold 3rd Degree Burns\strut
\end{minipage} & \begin{minipage}[t]{0.24\columnwidth}\raggedleft
1000\strut
\end{minipage}\tabularnewline
\bottomrule
\end{longtable}

To estimate the harm to an observer, the relationship between the
emitter (fireball) and the observer (person) must be determined. This
relationship is called the view factor and for a spherical emitter and
vertical surface can be found from,

\begin{equation}F_v = \frac{x(D/2)^2}{(x^2+H^2)^{3/2}}\end{equation}

where \(D\) is the diameter and \(H\) is the height to the center of the
fireball. The fireball is assumed to be spherical with a height of
\(H = 343.9ft\:(104.8m)\) and a diameter \(D = 312.9ft\:(95.4m)\). The
range \(x\) will be evaluated over a series of distances to determine
where critical points of human injury (pain and burns) are predicted to
occur, see \cref{fig:fig_fireball_thermal}.

\begin{figure}[H]
\hypertarget{fig:fig_fireball_thermal}{%
\begin{center}
\adjustimage{max size={0.9\linewidth}{0.9\paperheight},width=1.0\linewidth}{philadelphia_refinery_files/fig_radiation_estimate.png}
\end{center}
\caption{Diagram depicting the parameters for calculating the view factor used to
estimate the thermal radiation received by the receptor. The receptor
(human) is not shown to scale.}\label{fig:fig_fireball_thermal}
}
\end{figure}

Assuming a surface-emissive power of \(E = 350kW/m^2\) for the surface
of the fireball and an atmospheric transmissivity of \(\tau_a = 1\) it
is possible to estimate the radiation received by the observer using,

\begin{equation}q = EF_v\tau_a\end{equation}

\hypertarget{thermal-dose-units}{%
\subsection{Thermal Dose Units}\label{thermal-dose-units}}

Finally, the thermal dose units \((TDU)\) received by the observer can
be estimated by assuming they are exposed to the full fireball duration
of \(t_c = 14.80s\) using\cite{OSullivan2004},

\begin{equation}TDU = q^{4/3}t_c\end{equation}

If the range to the observer is varied from \(984.3ft\:(300m)\) to
\(3280.8ft\:(1000m)\) we can calculate the view factor \((F_v)\),
radiation \((q)\), and thermal dose units \((TDU)\) for the observer.
Refer to \cref{tbl:TDU} for a summary of the calculations. We can then
compare the critical \(TDU\) in \cref{tbl:tbl_TDU} to the calculated
\(TDUs\) at various ranges from the fireball, see \cref{fig:fig_tdu}.
For the observer to be exposed to 3rd degree burns they would need to be
inside of \(1312.3ft\:(400m)\), 2nd degree burns inside of
\(1968.5ft\:(600m)\), and 1st degree burns inside of
\(2952.76ft\:(900m)\).

\begin{table}[H]
\caption{Thermal Dose Units Delivered by the BLEVE to a Human in the Open}\label{tbl:TDU}
\centering
\begin{adjustbox}{max width=\textwidth}
\begin{tabular}{lrrr}
\toprule
{} &  View Factor &  Solid Flame Raditation ($kW/m^2$) &  Thermal Dose Units ($(kW/m^2)^{4/3}s$) \\
Ground Distance (m) &              &                                    &                                         \\
\midrule
300                 &       0.0850 &                            29.7609 &                               1364.9134 \\
350                 &       0.0653 &                            22.8472 &                                959.4496 \\
400                 &       0.0515 &                            18.0111 &                                698.7112 \\
450                 &       0.0415 &                            14.5238 &                                524.4272 \\
500                 &       0.0341 &                            11.9390 &                                403.8322 \\
550                 &       0.0285 &                             9.9761 &                                317.8266 \\
600                 &       0.0242 &                             8.4536 &                                254.8573 \\
650                 &       0.0207 &                             7.2506 &                                207.6883 \\
700                 &       0.0180 &                             6.2847 &                                171.6423 \\
750                 &       0.0157 &                             5.4980 &                                143.6089 \\
800                 &       0.0139 &                             4.8491 &                                121.4662 \\
850                 &       0.0123 &                             4.3078 &                                103.7339 \\
900                 &       0.0110 &                             3.8518 &                                 89.3576 \\
950                 &       0.0099 &                             3.4642 &                                 77.5724 \\
1000                &       0.0089 &                             3.1319 &                                 67.8145 \\
\bottomrule
\end{tabular}

\end{adjustbox}
\end{table}

\begin{figure}[H]\begin{center}\adjustimage{max size={0.9\linewidth}{0.9\paperheight},height=0.33\paperheight}{philadelphia_refinery_files/output_62_0.png}\end{center}\caption{The thermal dose units received by a human in the open if exposed to the
BLEVE fireball for 14.2s at a range of distances. The pain and 1st, 2nd,
and 3rd degree burn thresholds are indicated.}\label{fig:fig_tdu}\end{figure}

\hypertarget{conclusion}{%
\section{Conclusion}\label{conclusion}}

The BLEVE analysis indicates the following:

\begin{itemize}
\tightlist
\item
  The calculated overpressure \(28.66psia\:(197603.74Pa)\) at
  \(32.8ft\:(10m)\) would be lethal to a human in the open. The
  threshold for fatality would be approximately than \(869.4ft\:(265m)\)
\item
  The large and small-end-caps and the center-section debris had
  sufficient initial velocity \(318.85mph\:(142.54m/s)\) to be thrown
  the distances observed. The maximum throw distance possible would be
  \(6794.9ft\:(2071.1m)\).
\item
  The thermal effects would produce 2nd degree burns at
  \(1870ft\:(570m)\) and 3rd degree burns at \(1135ft\:(346m)\).
\end{itemize}

This BLEVE analysis is based on safety-engineering models therefore,
estimates for the consequences are considered conservative. This type of
analysis is used during the design phase of a plant to identify process
engineering hazards for employees and the public.

\bibliographystyle{unsrtnat}
\bibliography{philadelphia_refinery_files/library}

\end{document}
